% Vorlesung vom 16.10.2015
\renewcommand{\ldate}{2015-10-16}

\subsection{Ellipse}
\includegraphicsdeluxe{Ellipse.jpg}{Ellipse}{Eine Ellipse mit $a=3$ cm, $b=2$ cm, Punktkoordinaten}{fig:ellipse1}
Mit $ f(t)= \left( \begin{array}{c} a \cos(t)\\ b \sin(t) \end{array} \right), a,b > 0 $ (Halbachsen) wird eine Ellipse beschrieben (Abb. \ref{fig:ellipse1}). In unserem Beispiel sind $a=3$ cm und $b=2$ cm. 

\subsection{Geschwindigkeitsvektor}
\includegraphicsdeluxe{Geschwindigkeitsvektor.jpg}{Geschwindigkeitsvektoren}{Geschwindigkeitsvektor links: Richtung, |v| Betrag der Geschwindigkeit. Rechts: Durchschnittsgeschwindigkeitsvektor}{fig:geschwindigkeitsvektor1}
Wir berechnen den Geschwindigkeitsvektor [$t\widehat{=}$ Zeit]. Der Durchschnittsgeschwindigkeitsvektor berechnet sich durch $ \frac{f(t+h)-f(t)}{h}$, Test: $ f(t)+h\cdot \frac{f(t+h)-f(t)}{h} = f(t+h)$. (Abb. \ref{fig:geschwindigkeitsvektor1})

\paragraph{Momentangeschwindigkeitsvektor} 
$\lim\limits_{h\rightarrow 0} \frac{f(t+h)-f(t)}{h} \underbrace{=}_{\R^2} $ 
$\lim\limits_{h\rightarrow 0} \left(
\begin{array}{c}
\frac{x(t+h)-x(t)}{h}\\
\frac{y(t+h)-y(t)}{h}
\end{array}
\right) = \left( \begin{array}{c} \dot{x}(t) \\ \dot{y}(t) \end{array}\right) $ mit
$ \dot{x}(t) = \frac{dx}{dt}, \dot{y}(t) = \frac{dy}{dt} $. Der Ableitungsvektor ist der Geschwindigkeitsvektor. Er ist auch Tangentenvektor an die Kurve. 

\subsection{Beispiele}
\paragraph{1.} $ f(t) = \left( \begin{array}{c} t^3 + t\\ 2t\\t^2\end{array}\right), f'(t)= \left( \begin{array}{c} 3t^2 + 1\\2\\2t\end{array}\right)$
\paragraph{2. Kreis} 
$ f(t) = \left( \begin{array}{c} R \cos(t)\\ R\sin (t)\end{array}\right), $
$f'(t) = \left( \begin{array}{c} - R \sin(t)\\ R \cos(t)\end{array}\right)$\\
$R=1: f(t) = \left( \begin{array}{c} \cos(t)\\ \sin (t)\end{array}\right) $
$f'(1) = \left( \begin{array}{c} - \sin(t)\\ \cos(t)\end{array}\right)$ \\
$\abs{f'(t)} = \sqrt{\sin^2 (t) + \cos^2(t)}=1 $\\
Skalarprodukt: 
$\left( \begin{array}{c} \cos t\\\sin t\end{array}\right) \cdot \left( \begin{array}{c} -\sin t\\ cos t\end{array}\right) $
$= -\cos t \sin t + \sin t \cos t = 0 \Rightarrow 90^\circ $.
\includegraphicsdeluxe{geschwindigkeitsvektor_kreis.jpg}{Geschwindigkeitsvektor im Kreis}{Geschwindigkeitsvektor im Kreis}{fig:geschwindigkeitsvektor2}

\paragraph{Ellipse}
$ f(t)=\left( \begin{array}{c} a \cos t\\ b \sin t \end{array}\right), f'(t)=\left( \begin{array}{c} -a \sin t\\ b \cos t\end{array}\right)$

\subsection{Der Beschleunigungsvektor}
\includegraphicsdeluxe{beschleunigungsvektor.jpg}{Beschleunigungsvektor}{Der Beschleunigungsvektor}{fig:beschleunigungsvektor1}
Durchschnittsbeschleunigung (Abb. \ref{fig:beschleunigungsvektor1}) zwischen t und t+h: 
$ \frac{v(t+h)-v(t)}{h}$\\
Test: $ v(t)+ h\cdot \frac{v(t+h)-v(t)}{h} = v(t+h)$\\
Momentanbeschleunigung: 
$ b(t)=\lim\limits_{h\rightarrow 0} \frac{v(t+h)-v(t)}{h} = v'(t)$\\
$ b(t)=f''(t)$

\subsection{Merke}
\begin{enumerate}
\item Die erste Ableitung $f'(t)$ entspricht dem Geschwindigkeitsvektor $f'(t) = v(t)$.
\item Die zweite Ableitung $f''(t)$ entspricht dem Beschleunigungsvektor $f''(t)= v'(t)=b(t)$.
\end{enumerate}

\subsection{Beispiele}
\paragraph{1.} 
$ f(t)= \left( \begin{array}{c} 3t^2\\t^3\end{array}\right), v(t)=\left( \begin{array}{c} 6t\\3t^2 \end{array}\right), b(t)=\left( \begin{array}{c} 6\\ 6t\end{array}\right)$
\paragraph{2. Kreis} 
\includegraphicsdeluxe{beschleunigung_kreis.jpg}{Beschleunigung im Kreis}{Beschleunigung im Kreis}{fig:beschleunigung_kreis} 
Jetzt sehen wir uns die Beschleunigung im Kreis an (Abb. \ref{fig:beschleunigung_kreis}).

$f(\varphi) = \left( \begin{array}{c} \cos \varphi\\ \sin \varphi \end{array}\right)$\\
$v(\varphi) = \left( \begin{array}{c} -\sin \varphi\\ \cos \varphi \end{array}\right), \abs{v}=1 $\\
$b(\varphi)= \left( \begin{array}{c} -\cos \varphi\\ -\sin \varphi \end{array}\right), \abs{b}=1$\\
Zentrifuge: $R=5$ m, 2 Umdrehungen pro Sekunde, $\varphi=4\pi t = \varphi(t)$\\
$f(t)=\left( \begin{array}{c} R\cos \varphi\\ R\cos \varphi \end{array}\right) $
$= \left( \begin{array}{c} 5 \cos(4\pi t)\\ 5 \sin(4\pi t)\end{array}\right)$\\
$f'(t) = \left( \begin{array}{c} -5\cdot 4\pi \sin(4\pi t)\\ 5\cdot 4 \pi \cos(4\pi t) \end{array}\right)$\\
$f''(t) = -80\pi^2 \cdot \underbrace{\left( \begin{array}{c} \cos(4\pi t)\\ \sin(4\pi t) \end{array}\right)}_{\textrm{Länge 1}}$\\
$\abs{f''(t)}=80\pi^2 = 789 \frac{m}{s^2}$
\paragraph{Ellipse}
$ f(\varphi)=\left( \begin{array}{c} a \cos \varphi \\ b \sin \varphi \end{array}\right)$\\
$ f'(\varphi)= \left( \begin{array}{c} -a \sin \varphi\\ b \cos \varphi\end{array}\right)=v(\varphi)$\\
$ f''(\varphi)= \left( \begin{array}{c} -a \cos \varphi\\ - \sin \varphi\end{array}\right)=b(\varphi)$

