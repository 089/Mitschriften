% Vorlesung vom 17.12.2015
\renewcommand{\ldate}{2015-12-17}

\subsection{Beispiel}
$y_1' = -y_2$

$y_2' = y_1 + x$

$\vektor{y_1'\\y_2'} = \vektor{0 &  -1\\1 & 0} \vektor{y_1\\y_2} + \vektor{0\\x}$

$y' = A\cdot y + b$

Lösungen des homogenen Systems: 

$ \varphi_1(x) = \vektor{\cos x \\ \sin x}, \varphi_2(x)=\vektor{-\sin x\\\cos x}$

Test mit $\varphi_1$:

$\varphi_1'(x) = \vektor{-\sin x\\\cos x}, A_{\varphi_1} = \vektor{0,-1\\1,0} \vektor{\cos x\\\sin x} = \vektor{-\sin x\\\cos x}$

$\varphi_1, \varphi_2$ sind linear unabhängig, bilden also ein Fundamentalsystem.\index{Fundamentalsystem}

$\Phi(x) = \vektor{\cos x &  -\sin x\\\sin x &  \cos x}$. Was ist $\Phi^{-1}$? $\Phi$ ist orthogonal! 
$\Rightarrow \Phi^{-1} = \Phi^t = \vektor{\cos x &  \sin x\\-\sin x &  \cos x}$
$\Rightarrow u(x) = \int_{x_0}^{x} \vektor{\cos t & \sin t\\-\sin t &  \cos t} \vektor{0\\t} dt + \vektor{c_1\\c_2}$
$=\vektor{\int_{x_0}^{x} t \sin t dt + c_1\\\int_{x_0}^{x} t \cos t dt + c_2}$
$=\vektor{\sin x - x \cos x + d_1\\\cos x + x \sin x + d_2}$ \profnote{Die beiden Konstanten fassen wir zusammen: $d_1, d_2$}

Wähle ohne die Konstanten $u(x) = \vektor{\sin x - x \cos x\\\cos x + x \sin x}$
$\Rightarrow \Psi(x) = \Phi \cdot u$
$=\vektor{\cos x & -\sin x\\\sin x & \cos x} \cdot \vektor{\sin x - x \cos x\\\cos x + x \sin x}$
$=...=\vektor{-x\\1} = \Psi(x)$\\

\textbf{Alle Lösungen:}\\
$\Psi(x) + L_h = \vektor{-x\\1} + c_1 \cdot \vektor{\cos x \\ \sin x} + c_2 \cdot \vektor{-\sin x\\\cos x}, c_1,c_2 \in \R$
