% Vorlesung vom 16.10.2015
\renewcommand{\ldate}{2015-10-16}

\subsection{Ellipse}
\includegraphicsdeluxe{Ellipse.jpg}{Ellipse}{Eine Ellipse mit $a=3$ cm, $b=2$ cm, Punktkoordinaten}{fig:ellipse1}
Mit $ f(t)= \vektor{ a \cos(t)\\ b \sin(t) }, a,b > 0 $ (Halbachsen) wird eine Ellipse beschrieben (Abb. \ref{fig:ellipse1}). In unserem Beispiel sind $a=3$ cm und $b=2$ cm. 

\subsection{Geschwindigkeitsvektor}
\includegraphicsdeluxe{Geschwindigkeitsvektor.jpg}{Geschwindigkeitsvektoren}{Geschwindigkeitsvektor links: Richtung, |v| Betrag der Geschwindigkeit. Rechts: Durchschnittsgeschwindigkeitsvektor}{fig:geschwindigkeitsvektor1}
Wir berechnen den Geschwindigkeitsvektor [$t\widehat{=}$ Zeit]. Der Durchschnittsgeschwindigkeitsvektor berechnet sich durch $ \frac{f(t+h)-f(t)}{h}$, Test: $ f(t)+h\cdot \frac{f(t+h)-f(t)}{h} = f(t+h)$. (Abb. \ref{fig:geschwindigkeitsvektor1})

\paragraph{Momentangeschwindigkeitsvektor} 
$\lim\limits_{h\rightarrow 0} \frac{f(t+h)-f(t)}{h} \underbrace{=}_{\R^2} $ 
$\lim\limits_{h\rightarrow 0} \vektor{ \frac{x(t+h)-x(t)}{h}\\ \frac{y(t+h)-y(t)}{h} } = \vektor{ \dot{x}(t) \\ \dot{y}(t) } $ mit
$ \dot{x}(t) = \frac{dx}{dt}, \dot{y}(t) = \frac{dy}{dt} $. Der Ableitungsvektor ist der Geschwindigkeitsvektor. Er ist auch Tangentenvektor an die Kurve. 

\subsection{Beispiele}
\paragraph{1.} $ f(t) = \vektor{ t^3 + t\\ 2t\\t^2}, f'(t)= \vektor{ 3t^2 + 1\\2\\2t}$
\paragraph{2. Kreis} 
$ f(t) = \vektor{ R \cos t\\ R\sin t}, $
$f'(t) = \vektor{ - R \sin t\\ R \cos t}$\\
$R=1: f(t) = \vektor{ \cos t \\ \sin t} $
$f'(t) = \vektor{ - \sin t \\ \cos t}$ \\

\textbf{Aus der Formelsammlung: }\profnote{Für das Skalarprodukt wird üblicherweise $\circ$ oder $\cdot$ verwendet.}
\begin{enumerate}
\item $\sin^2 t + \cos^2 t =1$ 
\item $\abs{\vec{a}} = \sqrt{\vec{a} \circ \vec{a}}$
\end{enumerate}

$ \abs{f'(t)} 
= \sqrt{f'(t) \circ f'(t)}
= \sqrt{\vektor{ - \sin t \\ \cos t } \circ \vektor{ - \sin t\\ \cos t}}
= \sqrt{ (-\sin t) \cdot (-\sin t) + \cos t \cdot \cos t }
= \sqrt{\sin^2 t + \cos^2 t }=1 
$\\

\textbf{Skalarprodukt: }
$\vektor{ \cos t\\\sin t} \cdot \vektor{ -\sin t\\ cos t} $
$= -\cos t \sin t + \sin t \cos t = 0 \Rightarrow 90^\circ $.
\includegraphicsdeluxe{geschwindigkeitsvektor_kreis.jpg}{Geschwindigkeitsvektor im Kreis}{Geschwindigkeitsvektor im Kreis}{fig:geschwindigkeitsvektor2}

\paragraph{Ellipse}
$ f(t)=\vektor{ a \cos t\\ b \sin t }, f'(t)=\vektor{ -a \sin t\\ b \cos t}$

\subsection{Der Beschleunigungsvektor}
\includegraphicsdeluxe{beschleunigungsvektor.jpg}{Beschleunigungsvektor}{Der Beschleunigungsvektor}{fig:beschleunigungsvektor1}
Durchschnittsbeschleunigung (Abb. \ref{fig:beschleunigungsvektor1}) zwischen t und t+h: 
$ \frac{v(t+h)-v(t)}{h}$\\
Test: $ v(t)+ h\cdot \frac{v(t+h)-v(t)}{h} = v(t+h)$\\
Momentanbeschleunigung: 
$ b(t)=\lim\limits_{h\rightarrow 0} \frac{v(t+h)-v(t)}{h} = v'(t)$\\
$ b(t)=f''(t)$

\subsection{Merke}
\begin{enumerate}
\item Die erste Ableitung $f'(t)$ entspricht dem Geschwindigkeitsvektor $f'(t) = v(t)$.
\item Die zweite Ableitung $f''(t)$ entspricht dem Beschleunigungsvektor $f''(t)= v'(t)=b(t)$.
\end{enumerate}

\subsection{Beispiele}
\paragraph{1.} 
$ f(t)= \vektor{ 3t^2\\t^3}, v(t)=\vektor{ 6t\\3t^2 }, b(t)=\vektor{ 6\\ 6t}$
\paragraph{2. Kreis} 
\includegraphicsdeluxe{beschleunigung_kreis.jpg}{Beschleunigung im Kreis}{Beschleunigung im Kreis}{fig:beschleunigung_kreis} 
Jetzt sehen wir uns die Beschleunigung im Kreis an (Abb. \ref{fig:beschleunigung_kreis}).

$f(\varphi) = \vektor{ \cos \varphi\\ \sin \varphi }$\\
$v(\varphi) = \vektor{ -\sin \varphi\\ \cos \varphi }, \abs{v}=1 $\\
$b(\varphi)= \vektor{ -\cos \varphi\\ -\sin \varphi }, \abs{b}=1$\\
Zentrifuge: $R=5$ m, 2 Umdrehungen pro Sekunde, $\varphi=4\pi t = \varphi(t)$\\
$f(t)=\vektor{ R\cos \varphi\\ R\cos \varphi } $
$= \vektor{ 5 \cos(4\pi t)\\ 5 \sin(4\pi t)}$\\
$f'(t) = \vektor{ -5\cdot 4\pi \sin(4\pi t)\\ 5\cdot 4 \pi \cos(4\pi t) }$\\
$f''(t) = -80\pi^2 \cdot \underbrace{\vektor{ \cos(4\pi t)\\ \sin(4\pi t) }}_{\textrm{Länge 1}}$\\
$\abs{f''(t)}=80\pi^2 = 789 \frac{m}{s^2}$
\paragraph{Ellipse}
$ f(\varphi)=\vektor{ a \cos \varphi \\ b \sin \varphi }$\\
$ f'(\varphi)= \vektor{ -a \sin \varphi\\ b \cos \varphi}=v(\varphi)$\\
$ f''(\varphi)= \vektor{ -a \cos \varphi\\ - \sin \varphi}=b(\varphi)$

