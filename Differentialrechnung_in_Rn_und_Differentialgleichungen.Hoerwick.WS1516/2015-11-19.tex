% Vorlesung vom 19.11.2015
\renewcommand{\ldate}{2015-11-19}

\subsection{Lot auf eine Kurve}
\includegraphicsdeluxe{Hoehenlinienplan1.jpg}{Höhenlinienplan}{Höhenlinienplan: 0-er (rot)}{fig:Hoehenlinienplan1}
$f(x,y) = 0$, $f: \begin{cases}
\R^2 \rightarrow \R \\
(x,y) \rightarrow f(x,y)
\end{cases}$ ist ein \glqq Funktionsgebirge\grqq. Die Kurve $f(x,y)=0$ ist die 0-er Höhenlinie (Abb. \ref{fig:Hoehenlinienplan1}). Der Ableitungsvektor in P (Gradient) zeigt in Richtung des steilsten Anstieg. Das ist senkrecht zur Höhenlinie. Das Lot ist also der Gradient. Lot im Punkt (x,y) ist $ \rbr{\frac{\df}{\dx},\frac{\df}{\dy}}$ (rechts in Abb. \ref{fig:Hoehenlinienplan1}).

\subsection{Ellipse}
$f(x,y)=0$; $\frac{x^2}{a^2} + \frac{y^2}{b^2} - 1 = 0$; grad f $=\rbr{\frac{2x}{a^2}, \frac{2y}{b^2}}$ \profnote{grad ist der Gradient. }\\
Lot mit 45\grad:
$\frac{2x}{a^2} = \frac{2y}{b^2} \Rightarrow \frac{x}{a^2} = \frac{y}{b^2} \Rightarrow \frac{x^2}{a^4} = \frac{y^2}{b^4}$ (setzen wir unten ein)

$\frac{x^2}{a^2} + \frac{y^2}{b^2} -1 = 0 $
$\Rightarrow \frac{x^2}{a^4} \cdot a^2 + \frac{y^2}{b^2} -1 = 0 $
$\Rightarrow \frac{y^2}{b^4} \cdot a^2 + \frac{y^2}{b^2} = 1 $
$\Rightarrow y^2 \rbr{\frac{a^2}{b^4} + \frac{1}{b^2}} = 1 $
$\Rightarrow y^2 \frac{a^2 + b^2}{b^4} = 1 $
$\Rightarrow y^2 = \frac{b^4}{a^2 + b^2} $

z.B.: $a=3, b=2$, 
$y^2 = \frac{16}{9 + 4} $
$=\frac{16}{13} \Rightarrow y=1.1094$

$\frac{x}{a^2} = \frac{y}{b^2} $
$\Rightarrow x=\frac{a^2}{b^2} \cdot y$
$=\frac{9}{4} \cdot 1.1094 $
\underline{$=2.496$}

\subsubsection{andere Berechnungsmethode}

\textbf{Parameterdarstellung:} $(a \cos \varphi, b \sin \varphi)$

\textbf{Tangentenvektor:} $(- a \sin \varphi, b \cos \varphi)$

\textbf{Lotvektor:} $(b \cos \varphi, a \sin \varphi)$

\textbf{45\grad:} $b\cos \varphi = a\sin \varphi$ 
$\Rightarrow \frac{b}{a} $
$=\frac{\sin \varphi}{\cos \varphi}$
$=\tan \varphi$

Für $b=2, a=3: \frac{2}{3}=\tan \varphi \Rightarrow \varphi=33,69\gradi$

einsetzen: $(a \cos \varphi, b\sin \varphi)$
$=(3\cos 33.69\gradi, 2\sin 33.69\gradi)$
$=(2.496, 1.109)$

\subsection{Implizite Darstellung von Flächen im $\R^3$}
\includegraphicsdeluxe{ImplDarstVonFlaeR31.jpg}{Implizite Darstellung von Flächen im $\R^3$}{Implizite Darstellung von Flächen im $\R^3$}{fig:ImplDarstVonFlaeR31}

$f(x,y,z)=0$. Kugel um 0 mit dem Radius R: $x^2+y^2+z^2=R^2$.

\textbf{Drehellipsoid:} $\frac{x^2}{a^2} + \frac{z^2}{b^2} = 1$ (Abb. \ref{fig:ImplDarstVonFlaeR31})

\textbf{Rotation um Z-Achse:} $ \frac{r^2}{a^2} + \frac{z^2}{b^2} = 1; r^2=x^2+y^2$ 
$\Rightarrow \frac{x^2+y^2}{a^2} + \frac{z^2}{b^2} = 1$

\textbf{Lot in einem Flächenpunkt:}
$F: f(x,y,z) = 0, f:\R^3 \rightarrow \R$. Fläche F ist die \glqq 0-er Niveaufläche\grqq.

Bewegt man sich im Flächenpunkt P in Richtung grad(f), so wächst die Funktion am stärksten. Der Gradient in P ist also das Lot auf die Fläche in P (Abb. \ref{fig:ImplDarstVonFlaeR31}).

\subsection{Beispiel}

\textbf{Kugel:} $x^2+y^2+z^2-R^2 = 0 \Rightarrow grad(f) = (2x,2y,2z)$

\textbf{Drehellipsoid} $\frac{x^2}{a^2} + \frac{y^2}{a^2} + \frac{z^2}{b^2} -1 = 0$
$\Rightarrow grad(f) = \rbr{\frac{2x}{a^2}, \frac{2y}{a^2}, \frac{2z}{b^2}}$

\section{Die komplexen Zahlen}
\includegraphicsdeluxe{DieKomplZahlen1.jpg}{Komplexen Zahlen}{Menge der komplexen Zahlen. Menge M und Menge $\C$ (rot)}{fig:DieKomplZahlen1} 
Wir starten mit den reellen Zahlen $(\R,+,\cdot)$. $(M,+,\cdot)$ sei eine Erweiterung von $(\R,+,\cdot)$ (vgl. Abb. \ref{fig:DieKomplZahlen1}). In $(M,+,\cdot)$ sollen die üblichen Rechenregeln gelten. Das heißt, die Körperaxiome sollen gelten:

\begin{itemize}\profnote{Einschub}
\item $(M,+)$ kommutative Gruppe, 0 neutral
\item $(M^*,\cdot)$ kommutative Gruppe, 1 neutral
\item das Distributivgesetz muss gelten: $a(b+c) = ab+ac$
\end{itemize}

Wir nehmen an, es gibt ein $i\in M$ mit $i^2=-1$.

\begin{defi}
Wir definieren.: $\C = \cbr{x\in M: x=a+b\cdot i; a,b\in \R}$
\end{defi}

$(a+b i) + (c+d i) = (a+c) + (b+d)i \in \C$

$(a+b i) \cdot (c+d i) = (a c + a d i + b c i - b d)$ \profnote{Erinnerung: $i^2=-1$}
$=(ac-bd) + (ad+bc)i \in \C$

$\Rightarrow \C$ ist abgeschlossen bezüglich $+$ und $\cdot, \R \subset \C$. Die Darstellung $a+b i$ ist eindeutig. 
\begin{proof}
$a + bi = c + di$\\
$a-c = di-bi = (d-b)i$\\
angenommen: $d-b\neq 0 \Rightarrow i=\frac{a-c}{d-b}$ (!) \profnote{(!) Da schlägt der Blitz ein.}

Also: $d=b$ und $a=c$
\end{proof}

\begin{defi}
Wir definieren: $\C = \cbr{x\in M: x=a+b\cdot i; a,b\in \R}$

\textbf{$+$}: $(a,b)+(c,d)=(a+c,b+d)$

\textbf{$\cdot$}: $(a,b)\cdot (c,d)=(a c - bd,ad+bc)$
\end{defi}

Die reellen Zahlen sind in $\C$ eingebettet. 
$x\in \R \leftrightarrow (x,0) \in \C$

$(x,0) + (y,0) = (x+y,0)$

$(x,0) \cdot (y,0) = (xy,0+0)$\\

In $(\C,+,\cdot)$ gelten die Körperaxiome:
\begin{itemize}
\item assoziativ: $a+(b+c)=(a+b)+c$, $a\cdot (b\cdot c)=(a\cdot b)\cdot c$
\item kommutativ: $a+b=b+a$, $a\cdot b=b\cdot a$
\item distributiv: $a (b+c)= ab+ac$
\end{itemize}

$(0,0)$ ist neutrales Element bezüglich der Addition, $(1,0)$ bezüglich der Multiplikation. Inverses Element bezüglich der Addition: $(a,b)+(-a,-b)=(0,0)$. Inverses Element bezüglich der Multiplikation: $(a,b)\cdot (\frac{a}{a^2+b^2},\frac{-b}{a^2+b^2})=(1,0)$. 
Schreibweisen: $x+(-y) = x-y$ und $x\cdot y^{-1} = \frac{x}{y}$
