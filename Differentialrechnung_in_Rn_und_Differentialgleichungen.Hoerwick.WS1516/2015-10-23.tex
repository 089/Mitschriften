% Vorlesung vom 23.10.2015
\renewcommand{\ldate}{2015-10-23}

\subsection{Bogenlänge der Zykloide}
$f(\varphi)=R \rbr{\varphi - \sin \varphi, 1 - \cos \varphi}, $
$f'(\varphi)= R \rbr{1-\cos \varphi, \sin \varphi} $\\
$\abs{f'(\varphi)} = R \sqrt{(1-\cos \varphi)^2 + sin^2 \varphi}$
$= R \sqrt{1+\cos^2 \varphi - 2 \cos \varphi + \sin^2 \varphi}$
$= R \sqrt{2-2\cos \varphi}$
$= 2 R \sqrt{\frac{2-2\cos \varphi}{4}}$
$= 2 R \sqrt{\frac{1- \cos \varphi}{2}}$
$\underbrace{=}_{\textrm{Formelsammlung}} 2R \sin \frac{\varphi}{2}$\\
$\int_{0}^{2\pi} \abs{f'(\varphi)} d\varphi$
$=\int_{0}^{2\pi} 2 R \sin \frac{\varphi}{2} d\varphi$
$= \sbr{-4 R \cos \frac{\varphi}{2}}_{0}^{2\pi}$
$= - 4 R \rbr{\cos \pi - \cos 0}$
$=- 4 R (-1-1)$
\underline{$=8 r$} (Länge), \underline{Weg: $2\pi R$}

\subsection{Die natürliche Parameterdarstellung}

\begin{defi}
Eine Parameterdarstellung k(t) heißt natürlich, wenn $ \abs{k'(t)=1}, \forall t $ (konstante Geschwindigkeit 1).
\end{defi}
\includegraphicsdeluxe{natParamDar1.jpg}{Parametrisierung nach der Bogenlänge}{Parametrisierung nach der Bogenlänge. Eine Parameterdarstellung ist natürlich, wenn $L_{a,t} = t-a$}{fig:natParamDar1}
Sei k(t) natürlich: 
$L_{a,b}$
$= \int_{a}^{t_0} \abs{f'(t)} dt$
$= \int_{a}^{t_0} 1 dt$
$= \sbr{t}_{a}^{t_0}$
\underline{$= t_0 -a$}

\subsection{Parametertransformation}
\includegraphicsdeluxe{paramTrans1.jpg}{Parametertransformation}{Parametertransformation}{fig:paramTrans1}
Die Funktion $t(\theta)$ sei streng monoton wachsend (in Abb. \ref{fig:paramTrans1} ist das so!) oder streng monoton fallend. Neue Parameterdarstellung der Kurve: $k(t(\theta)), \theta \in \sbr{c,d}.$\\
$t(\theta)$ steng monoton wachsend: Durchlaufsinn bleibt (gestrichelten Linien in Abb. \ref{fig:paramTrans1}).\\
$t(\theta)$ steng monoton fallend: anderer Durchlaufsinn.

\subsection{Beispiel}
$f(t) = \vektor{\cos t\\ \sin t}, t\in \sqrt{0,2\pi}$\\
$f(\theta) = 2\theta + 1$ (monoton steigend)\\
$k(t(\theta)) = \vektor{\cos(2\theta +1)\\ \sin(2\theta+1)}$\\
$t(c) = 0, 2c+1=0 \Rightarrow c=\frac{1}{2}$\\
$t(d) = 2\pi, 2d+1=2\pi \Rightarrow d=\frac{2\pi -1}{2}$

\subsection{Umwandlung einer Parameterdarstellung in eine natürliche Parameterdarstellung}
\includegraphicsdeluxe{umwandlParam2natParam.jpg}{Umwandlung einer Parameterdarstellung }{Umwandlung einer Parameterdarstellung in eine natürliche Parameterdarstellung. Die Bogenlänge soll so lang sein wie die Zeit (rote Linien).}{fig:umwandlParam2natParam}
Wir wollen eine Parameterdarstellung in eine natürliche Parameterdarstellung umwandeln (Abb. \ref{fig:umwandlParam2natParam}). 
$ \theta(t) - c = \int_{a}^{t} \abs{k'(s)} ds $. \profnote{Wir suchen das $ t(\theta) $, haben aber nur $ \theta(t) $.}
$ t(\theta) $ ist die Umkehrfunktion von $ \theta(t) $. 

\subsection{Beispiel Zykloide}
Wir suchen die natürliche Parameterdarstellung der Zykloide. 
$ R=1, f(t)=(t-\sin t, 1- \cos t), a=0 \leq t \leq b=\pi$ (halber Bogen bis 180\grad). Setze $c=0 (d=4), \abs{f't(t)}=2 \sin \frac{t}{2}$ (Wir ersetzen im Folgenden t durch s).\\
$\theta(t)=\int_{0}^{t} 2 \sin \frac{s}{2} ds$
$= \sbr{4 \cos \frac{s}{2}}_{s=0}^{s=t}$
$=-4 \rbr{\cos \frac{t}{2} - \cos 0}$
$= -4 \cos \frac{t}{2} + 4$\\
$\theta(t) = 4-4\cos \frac{t}{2} = \theta $ Nun suchen wir die Umkehrfunktion $\to t(\theta)$ nach t auflösen. 
$-4 \cos \frac{t}{2} = \theta - 4$\\
$\cos \frac{t}{2} = 1 - \frac{\theta}{4}$\\
$\frac{t}{2} = \arccos\rbr{1-\frac{\theta}{4}}$\\
$t = 2 \arccos\rbr{1-\frac{\theta}{4}}, 0\leq \theta \leq 4$ \\
$t(\theta) = 2 \arccos \rbr{1-\frac{\theta}{4}}$\\
$f(t(\theta)) = \vektor{2 \arccos \rbr{1-\frac{\theta}{4}} - \sin \rbr{2 \arccos \rbr{1-\frac{\theta}{4}}} \\ 1-\cos \sbr{2 \arccos \rbr{1-\frac{\theta}{4}}}}$

\subsection{Die Krümmung einer Kurve}
\includegraphicsdeluxe{krKurve1.jpg}{Krümmung einer Kurve}{Die Krümmung einer Kurve}{fig:krKurve1}
Die Parameterdarstellung $k(t)$ sei \underline{\textbf{natürlich}}. \\
$ \textrm{Krümmung} = \frac{\textrm{Winkeländerung des Geschwindigkeitsvektors}}{\textrm{Bogenlänge}}$\\
$ \textrm{Krümmung} = \lim\limits_{h\rightarrow 0} \frac{\abs{k'(t+h) - k'(t)}}{\abs{h}} $ 
\profnote{Zähler $ \widehat{=} $ blaue Linie in Abb. \ref{fig:krKurve1}}
$ = \lim\limits_{h\rightarrow 0} \abs{\frac{k'(t+h) - k'(t)}{h}}$\\
Krümmungsvektor (ohne Betrag): $\lim\limits_{h\rightarrow 0} \frac{k'(t+h) - k'(t)}{h}$
\underline{$ = k''(t)$}\\
Krümmung: $\abs{k''(t)}$\\
Der Krümmungsvektor $k''(t) $ steht senkrecht auf $k'(t)$. 

\paragraph{Annäherung durch Kreis}
%\includegraphicsdeluxe{annaeherungKreis1.jpg}{Annäherung an eine Kurve}{Annäherung an die Kurve mit einem Kreis. $\alpha = \frac{h}{R} $ zeigt in Richtung des Kreismittelpunkt.}{fig:annaeherungKreis1}
Wir nähern die Kurve k im Punkt k(t) durch einen Kreis an (Abb. \ref{fig:krKurve1}). 
$\lim\limits_{h\rightarrow 0} \frac{k'(t+h) - k(t)}{h} $ mit $\alpha = \frac{h}{R}$\\
$\lim\limits_{h\rightarrow 0} \frac{\abs{k'(t+h) - k(t)}}{\abs{h}} $
$=\lim\limits_{h\rightarrow 0} \frac{\alpha}{\alpha \cdot R} $
$=\frac{1}{R}$. Der Betrag der Krümmung ist somit $\frac{1}{R}$. Also:\\
Der Krümmungsvektor zeigt in Richtung des Krümmungskreismittelpunkts. 
Der Krümmungskreisradius ist $\frac{1}{\textrm{Krümmung}}$. Die Ebene des Krümmungskreises wird aufgespannt durch den Geschwindigkeitsvektor und den Krümmungsvektor. 
