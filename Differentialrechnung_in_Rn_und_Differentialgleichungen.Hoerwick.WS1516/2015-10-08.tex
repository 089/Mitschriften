% Vorlesung vom 08.10.2015
\renewcommand{\ldate}{2015-10-08}

\subsection{Definition Gradient}
Wiederholung Richtungsableitung: $ f_v(x)=\lim\limits_{h\rightarrow 0} \frac{f(x+h\cdot v) - f(x)}{h}$
\begin{defi}
$ f: \R^n \rightarrow R $\\
Der Vektor $ \rbr{\frac{\delta f}{\delta x_1}, \frac{\delta f}{\delta x_2}, ..., \frac{\delta f}{\delta x_n} }$ heißt der Gradient von f bei x.
\end{defi}

\begin{satz}
$ f: \R^n \rightarrow \R $\\
$ v = (v_1, ..., v_n) \textrm{ mit } |v|=1. $
Dann gilt: $ f_v(x_1, ..., x_n) = (v_1, ..., v_n) \cdot \frac{\delta f}{\delta x_1}, \frac{\delta f}{\delta x_2}, ..., \frac{\delta f}{\delta x_n} $\\
$ f_v(x) = v \cdot \textrm{Gradient von f } (\widehat{=} \textrm{ Skalarprodukt: } (x_1,x_2)\cdot (y_1,y_2)=x_1y_1+x_2+y_2) $ 
\end{satz}

% \profnote{Jetzt machen wir einen Beweis. Wenn man den schlampig macht, geht's recht schnell. Wir machens schlampig.}

\begin{proof}
$ f(x_1+dx_1, ..., x_n+dx_n) \approx f(x_1, ..., x_n) + \frac{\delta f}{\delta x_1} \cdot dx_1 + ... + \frac{\delta f}{\delta x_n} \cdot dx_n  $\\
$ f(x_1+h v_1,x_2+h v_2, ..., x_n+h v_n) $
$ \approx f(x_1, ..., x_n) + \frac{\delta f}{\delta x_1} \cdot h\cdot v_1 + ... + \frac{\delta f}{\delta x_n} \cdot h\cdot v_n $\\
Einsetzen der Grenzwertbildung:\\
$ \lim\limits_{h\rightarrow 0} = \frac{f(x_1+hv_1,...,x_n+hv_n) - f(x_1, ..., x_n)}{h} $
$ = \lim\limits_{h\rightarrow 0} \frac{f(x_1,...x_n)+\frac{\delta f}{\delta x_1}hv_1 + ... + \frac{\delta f}{\delta x_n}hv_n - f(x_1, ..., x_n)}{h} $ 
$ = \lim\limits_{h\rightarrow 0} = \frac{\delta f}{\delta x_1} v_1 + ... + \frac{\delta f}{\delta x_n} v_n = \textrm{Gradient von f} \cdot v $
\end{proof}

\subsection{Beispiel von oben}
$ f(x_1,x_2,x_3) = x_1x_2 + 2x_3, v=(\frac{1}{3}, \frac{2}{3}, \frac{2}{3}) $\\
$ f_v(x_1,x_2,x_3)=(v_1,v_2,v_3)\cdot (\frac{\delta f}{\delta x_1}, \frac{\delta f}{\delta x_2}, \frac{\delta f}{\delta x_3}) $
$ = (\frac{1}{3}, \frac{2}{3}, \frac{2}{3})\cdot (x_2, x_1, 2) = \frac{1}{3} x_2 + \frac{2}{3} x_1 + \frac{4}{3} $

\subsection{Hausaufgabe}
% \profnote{Das können Sie zu Hause rechnen. }
$ f(x_1, x_2, x_3, x_4) = x_1^2 x_2 x_3 + x_3^2 x_4$\\
Richtung $ \tilde{v} = (1,-1,-1,1) $ \\
Richtungsableitung in: $ (1,0,2,-1) $

\subsection{Problem} $ f:\R^n \rightarrow \R $
Folgende Fragen stellen wir uns: 
\begin{itemize}

\includegraphicsdeluxe{steilsteRichtungsableitung.jpg}{Steilste Richtungsableitung}{In welcher Richtung geht es am steilsten bergauf?}{fig:steilste_richtungsableitung1}

\item In welcher Richtung v ist die Richtungsableitung am größten?
\item In welcher Richtung wächst die Funktion am schnellsten?
\item In welcher Richtung geht es am steilsten Bergauf? (Abb. \ref{fig:steilste_richtungsableitung1})
\end{itemize}

\includegraphicsdeluxe{gradientf.jpg}{Gradient von f}{Gradient von f}{fig:gradientf1} 

$ \cos = \frac{v gradf}{|v| \cdot |gradf|} = \frac{f_v}{|gradf|} \Rightarrow f_v=\cos \delta \cdot |gradf| $ (Abb. \ref{fig:gradientf1}). $f_v$ ist maximal bei $\cos \delta = 1$, d.h. bei $\delta=0^\circ$. Die Richtungsableitung ist maximal in Richtung gradf. Die maximale Richtungsableitung ist $|gradf|$. 

\subsection{Beispiel}
$ f(x_1,x_2,x_3) = 3x_1x_2 + 2x_3^2 $\\
Punkt: (1,2,-2)\\
In welcher Richtung wächst f am stärksten? \\
$ gradf = (3x_2, 3x_1, 4x_3) $\\
$ gradf(1,2,-2) = \underline{(6,3,-8)} $ (gesuchte Richtung)\\
Die maximale Steigung ist $ |gradf| = |(6,3,-8)| = \sqrt{36+9+64} = \sqrt{109} $\\
Steigungswinkel $\alpha = \arctan (\sqrt{109}) = 84,5^\circ $ 

\subsection{Hausaufgabe}
$ f(x_1,x_2,x_3,x_4) = x_1x_2^2 - x_3 x_4 + x_2 x_3^2$\\
Punkt: (2,1,-3,2)\\
In welcher Richtung wächst f am stärksten?\\
Wie groß ist dort der Steigungswinkel?

\paragraph{Lösung} //TODO

\section{Integration von Funktionen des Typs $ \R^2 \rightarrow \R $}
\includegraphicsdeluxe{zylinderVolumen.jpg}{Volumen eines beliebigen Körpers}{Das Volumen soll dem Integral $ \int_{B} f $ entsprechen. Dabei ist die Fläche B beliebig und der obere \textquote{Deckel} des \textquote{Zylinders} entspricht irgendeinem \textquote{Funktionsgebirge}}{fig:zylinderVolumen1}

$ \int_{B} f $ ist das Volumen des \textquote{Zylinders} über B (von B bis zur Funktion, vgl. Abb. \ref{fig:zylinderVolumen1}). 

\subsection{Annäherung} 
\includegraphicsdeluxe{annaeherungVolumen1.jpg}{Annäherung Volumen}{Zur Annäherung des Volumens betrachten wir nun die Draufsicht $ f:[a_1,a_2]\times [b_1,b_2] \rightarrow \R$}{fig:annaeherungVolumen1}

Zur Annäherung des Volumens betrachten wir nun die Draufsicht (Abb. \ref{fig:annaeherungVolumen1}).\\
Sei $ \tilde{f} : [a_1,a_2]\times [b_1,b_2] \rightarrow \R $\\
$ (x,y) \rightarrow f(x,y) \textrm{ für } (x,y) \in B, 0 \textrm{ sonst} $ 

\subsection{Annäherung durch Scheiben}
\includegraphicsdeluxe{annaeherungVolumen2.jpg}{Annäherung Volumen durch Scheiben}{Nun näheren wir uns dem Volumen durch eine Unterteilung in Scheiben (rot) an.}{fig:annaeherungVolumen2}

\includegraphicsdeluxe{annaeherungVolumen3.jpg}{Querschnittsfläche bei x}{Querschnittsfläche bei x. Das Ergebnis ist irgendeine Funktion $ g(x) = \int_{b_1}^{b_2} f(x,y)dy $ }{fig:annaeherungVolumen3}

Nun näheren wir uns dem Volumen durch eine Unterteilung in Scheiben (rot) an (Abb. \ref{fig:annaeherungVolumen2}).

$ \int_{B} f(x,y) \approx \sum_{i=1}^{n} \underbrace{( \int_{b_1}^{b_2} \tilde{f} (z_i,y) dy ) \Delta x)}_{\textrm{Das Volumen als Summe der einzelnen Scheiben}} $
$ \rightarrow \sum_{i=1}^{n} g(z_i)\cdot \Delta x \overrightarrow{ _{n\rightarrow \inf}} \int_{a_1}^{a_2} g(x) dx $ 
$ = \int_{a_1}^{a_2} ( \int_{b_1}^{b_2} f(x,y)dy)dx $ (für $ g(z_i) $ vgl. Abb. \ref{fig:annaeherungVolumen3})\\
\textbf{also:}
$ \int_{B} f(x,y) = \int_{a_1}^{a_2} ( \int_{b_1}^{b_2} \tilde{f}(x,y)dy)dx $ (Doppelintegral über die gelben Querschnitte)\\

\textbf{analog:}
$ \int_{B} f(x,y) = \int_{b_1}^{b_2} ( \int_{a_1}^{a_2} \tilde{f}(x,y)dx ) dy $ (Doppelintegral über die roten Querschnitte)

\subsection{Beispiel} 
\includegraphicsdeluxe{beispiel1.jpg}{Beispiel}{Flächenberechnung durch Integration über die gelben oder roten Querschnitte}{fig:beispiel1}

$ f: \R^2 \rightarrow \R $\\
$ f(x,y) = 1+xy^2 $\\
$ \int_B f(x,y) $ (vgl. Abb. \ref{fig:beispiel1})\\
Als erstes integrieren wir über die gelben Querschnitte:\\
$ \int_{B} f(x,y) = \int_0^3 ( \int_0^2 (1+xy^2) dy ) dx $
$ = \int_0^3 (x + \frac{1}{3} xy^3)|_{y=0}^{y=2} dx = \int_0^3 (2+\frac{1}{3} x\cdot 8) dx $
$ = \int_0^3 (2+ \frac{8}{3} x) dx = 2x+\frac{8}{3} \cdot \frac{1}{2} x^2 |_{x=0}^{x=3} $
$ = 2\cdot 3 + \frac{4}{3} \cdot 9 = 6+12=18 $\\
Nun integrieren wir über die roten Querschnitte. Es muss das gleiche rauskommen: \\
$ \int_0^2 ( \int_0^3 f(x,y)dx)dy $
$ = \int_0^2 ( \int_0^3 (1+xy^2 dx) dy $
$ = \int_0^2 ( y+\frac{1}{2}x^2y^2)_{x=0}^{x=3} dy $
$ = \int_0^2 (3+\frac{1}{2}\cdot 9 y^2)dy $
$ = 3y + \frac{9}{2} \cdot \frac{1}{3} y^3 |_{y=0}^{y=2} $
$ 3\cdot 2 + \frac{3}{2} \cdot 8 = 6+12=18 $


