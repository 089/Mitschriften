% Vorlesung vom 12.11.2015
\renewcommand{\ldate}{2015-11-12}

\subsection{Beispiel}
$ f: \R \rightarrow \R^2, x \rightarrow \vektor{x \\ \cos x}$, 
$ g: \R^2 \rightarrow \R, \vektor{x\\y}\rightarrow x^y$

\textbf{gesucht:} $(g\circ f)'$
\begin{enumerate}
\item direkt
\item mit Kettenregel
\end{enumerate}

\paragraph{1) direkt} 
$ (g\circ f)(x) $
$= g\vektor{x\\ \cos x} $
$=x^{\cos x}$
$=h(x)$\\
$h(x) $
$= e^{\ln x \cdot \cos x}$, 
$h'(x) $
$= e^{\ln x \cdot \cos x} \cdot \rbr{\frac{1}{x} \cos x - \ln x \sin x}$
$=x^{\cos x} \cdot  \rbr{\frac{1}{x} \cos x - \ln x \sin x}$

\paragraph{2) mit der Kettenregel}
$(g\circ f)'(x)$
$=g'(f(x)) \cdot f'(x)$
mit $f(x) $
$= \vektor{x\\\cos x}$
$= \vektor{f_1(x)\\f_2(x)}$
$\Rightarrow f'(x) = \vektor{1\\-\sin x}$
\marginpar{Aus der Formelsammlung:\\ $x^n \rightarrow n x^{n-1}, a^y \rightarrow a^y \cdot \ln a$}

$g\vektor{x\\y}=x^y$
$\Rightarrow g'\vektor{x\\y}$
$=\rbr{y\cdot x^{y-1}, x^y\cdot \ln x}$

$g'(f(x)) $
$=g'\vektor{x\\\cos x}$
$=\rbr{\cos x\cdot x^{(\cos x)-1}, x^{\cos x} \cdot \ln x}$

$g'(f(X)) \cdot f'(x)$
$=\rbr{\cos x\cdot x^{(\cos x)-1}, x^{\cos x} \cdot \ln x} \cdot \vektor{1\\-\sin x}$
$=\cos x\cdot x^{(\cos x)-1} - x^{\cos x} \ln x \sin x $
$=x^{\cos x} \cdot  \rbr{\frac{1}{x} \cos x - \ln x \sin x}$

\section{Parameterdarstellung von Flächen}
\includegraphicsdeluxe{ParamDarVonFla.jpg}{Parameterdarstellung von Flächen}{Das rote ist die $\lambda$-Parameterlinie (Immer die Variable), das blaue die $\mu$-Parameterlinie}{fig:ParamDarVonFla}
$f:F \subset \R^2 \rightarrow \R^3$\\
$\vektor{\lambda\\\mu} \rightarrow \vektor{f_1(\lambda & \mu)\\f_2(\lambda & \mu)\\f_3(\lambda & \mu)}$

\subsection{Beispiel}
\includegraphicsdeluxe{BspErdkugel1.jpg}{Beispiel Erdkugel}{Beispiel Erdkugel: Erdachse: $z=R\sin \varphi $, Breite: $ -\frac{\pi}{2} \leq \varphi \frac{\pi}{2} $, Länge: $ 0\leq \lambda \leq 2\pi $}{fig:BspErdkugel1}
Parameterdarstellung: Wir betrachten die Erdkugel um 0 mit Radius R.\\
$z=R\cos \varphi$\\
$x=R\cos \varphi$\\
$y=R\cos \varphi \sin \lambda$\\
$\lambda$-Parameterlinie ist der Breitenkreis, $\varphi$-Parameterlinie der Längenhalbkreis.

\subsection{Parameterdarstellung des Drehellipsoids}
\includegraphicsdeluxe{ParamDrehellipsoids1.jpg}{Parameterdarstellung des Drehellipsoids}{Parameterdarstellung des Drehellipsoids: $x=a\cos \varphi, z=b\sin \varphi$}{fig:ParamDrehellipsoids1}
Wir lassen eine Ellipse um ihre Nebenachse rotieren (Abb. \ref{fig:ParamDrehellipsoids1}). 
\profnote{Wir konstruieren die Ellipse wie in Abb. \ref{fig:senkrDreiecke1}} 

$r=a\cos \varphi$\\
$x=r\cos \lambda$\\
$y=r \sin \lambda$\\
$\Rightarrow$ Parameterdarstellung\\
$x=a \cos \varphi \cos \lambda$\\
$y=a\cos \varphi \sin \lambda$\\
$z=b\sin \varphi$\\

$\varphi$-Parameterlinie: Längenhalbellipse, Länge $\lambda$\\
$\lambda$-Parameterlinie: Breitenkreis, die geographische Breite ist nicht $\varphi$ sondern $\tilde{\varphi}$.

\section{Implizite Darstellung von Kurven und Flächen}

\subsection{Kurven in $\R^2$}
$f(x,y) = 0$ Kreis um den Mittelpunkt $\vektor{a\\b}$ mit Radius R. \\
Distanz $d\rbr{\vektor{x\\y},\vektor{a\\b}^2} = R^2$\\
$\abs{\vektor{x-a\\y-b}}^2 = R^2$\\
$(x-a)^2 + (y+b)^2 = R^2$

\textbf{Gerade:} $ax+by+c = 0$

\textbf{Ellipse:} $\frac{x^2}{a^2} + \frac{y^2}{b^2} = 1$ mit a,b Halbachsen. 

\textbf{Sonderfall Kreis:} $\frac{x^2}{R^2} + \frac{y^2}{R^2} = 1$
$\Rightarrow x^2+y^2=R^2 \checkmark$

