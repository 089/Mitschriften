% Vorlesung vom 30.11.2015
\renewcommand{\ldate}{2015-11-30}

\subsection{Summe von zwei unabhängigen Zufallsgrößen}
$P(X=+Y=u)$
$=\sum_{i=1}^{r} \sum_{j=1}^{s} P(X=x_i, Y=y_j)$ mit $x_i + y_i = u$
$=\sum_{i=1}^{r} \sum_{j=1}^{s} P(X=x_i) \cdot P(Y=y_j)$ mit $x_i + y_i = u$

\subsection{Beispiel} 
Zweimal würfeln mit X erste Augenzahl, Y zweite Augenzahl. 
Wertebereich von $X+Y$: 2,3,...,12

$P(X+Y = u)$
$=\sum_{i=1}^{6} \sum_{j=1}^{6} \frac{1}{6} \frac{1}{6}$ mit $i+j=u$ 

$P(X+Y=10) = \frac{1}{36}$ mit $i+j=10$, also 4 6, 5 5 und 6 4 $\Rightarrow$ kommt drei mal vor
$=\frac{1}{36} \cdot 3 $
$=\frac{3}{36}$

\begin{tabular}{|c|c|c|c|c|c|c|c|c|c|c|c|}
\hline $X+Y$ & 2 & 3 & 4 & 5 & 6 & 7 & 8 & 9 & 10 & 11 & 12 \\ 
\hline P & $\frac{1}{36}$ & $\frac{2}{36}$ & $\frac{3}{36}$ & $\frac{4}{36}$ & $\frac{5}{36}$ & $\frac{6}{36}$ & $\frac{5}{36}$ & $\frac{4}{36}$ & $\frac{3}{36}$ & $\frac{2}{36}$ & $\frac{1}{36}$ \\ 
\hline 
\end{tabular} 

$E(X+Y)$
$=\frac{1}{36} \rbr{2\cdot 1 +  3\cdot 2+4\cdot 3+5\cdot 4+6\cdot 5+7\cdot 6+8\cdot 5+9\cdot 4+10\cdot 3 + 11\cdot 2 + 12\cdot 1}$
$=7$

Das geht auch einfacher, denn den Erwartungswert kann man auch zerlegen. Bei + ist es, im Gegensatz zu $\cdot$, auch egal ob die Zufallsgrößen unabhängig sind oder nicht: 
$E(X+Y) = EX + EY$
$=3.5+3.5$
$=7$

\subsection{Standardmodell}
Wir haben zwei Zufallsexperimente: $(\Omega_1,P_1), (\Omega_2,P_2)$

Zusammen: $\Omega = \Omega_1 \times \Omega_2$, 
$P(\cbr{\omega}) = P(\cbr{\omega_1, \omega_2})$
$=P_1(\cbr{\omega_1}) \cdot P_2(\cbr{\omega_2})$

$X: \Omega \rightarrow \R, (\omega_1, \omega_2) \rightarrow ...$, hängt nur von $\omega_1$ ab.

$Y: \Omega \rightarrow \R, (\omega_1, \omega_2) \rightarrow ...$, hängt nur von $\omega_2$ ab.

Dann sind X und Y unabhängig. 

\section{Binomial- und Multinomialverteilung}
Ein Zufallsexperiment habe 2 Ausgänge (Treffer 1, Niete 0). Trefferwahrscheinlichkeit p, Nietenwahrscheinlichkeit $1-p=q$. So ein Experiment nennt man auch Bernoulli-Experiment. Das Experiment wird n mal wiederholt, was man Bernoullikette nennt. 

$P(\cbr{(\omega_1, \omega_2, ..., \omega_n)}) $
$=p\cdot p\cdot p\cdot q\cdot p\cdot ... \cdot q \cdot p$
$=p^{\sum_{i=1}^{n} \omega_i} \cdot q^{n - \sum_{i=1}^{n} \omega_i} $

$A_i$ Ereignis: im Versuch i ein Treffer.

$Y=\sum_{i=1}^{n} I\cbr{A_i}$ Zählvariable. I ist die Indikatorfunktion.

$Y(\omega_1, ..., \omega_n) = $ Anzahl der Einser. 

\paragraph{Exkurs: Binomialverteilung}
$P(Y=k) $
$= \binom n k p^k \cdot q^{n-k}$, $k=0,1,...,n$

Y ist binomialverteilt. 

$EY$
$=E\rbr{\sum_{i=1}^{n} J\cbr{A_i}}$
$=\sum_{i=1}^{n} E I\cbr{A_i}$
$=\sum_{i=1}^{n} P(A_i) $
$=\sum_{i=1}^{n} p $
\underline{$= n\cdot p$}

\underline{$EY = n\cdot p$}

\begin{satz}
$(\Omega, P)$ Wahrscheinlichkeitsraum. 
$A_1, A_2, ..., A_n$ unabhängige Ereignisse mit $P(A_i)=p$

Dann ist $Z=\sum_{j=1}^{n} I\cbr{A_j}$ binomialverteilt $Bin(n,p)$

$Z(\omega)$ ist die Anzahl der $A_j$, in denen $\omega$ ist. 
\end{satz}

\subsection{Verallgemeinerung zur Multinomialverteilung}
Ein Zufallsexperiment habe s Ausgänge. 

\begin{tabular}{|c|c|c|c|c|c|}
\hline  & 1 & 2 & 3 & ... & s \\ 
\hline Wahrscheinlichkeit & $p_1$ & $p_2$ & $p_3$ &  & $p_s$ \\ 
\hline 
\end{tabular} 

$p_1+P_2+...+p_s=1$. Es wird n mal wiederholt. 

$P(\cbr{(3,4,1,1,5)})$
$=p_3\cdot p_4\cdot p_1\cdot p_1\cdot p_5, n=5 $

$X_1(\omega): $ Anzahl der Einser in $\omega$. 

$X_2(\omega): $ Anzahl der Zweier in $\omega$.

$\vdots$

$X_s(\omega): $ Anzahl der s in $\omega$.

$X=(X_1, X_2, ..., X_s)$ ist multinomialverteilt. 

$P(X_1=i_1, X_2=i_2, ..., X_s=i_s)$
$=p_1^{i_1} \cdot p_2^{i_2} \cdot ... \cdot p_s^{i_s} $
$=\frac{n!}{i_1! \cdot i_2! \cdot ... \cdot i_s!}$

\subsection{Beispiel}
n Kugeln, in s Farben, $i_1$ rot, $i_2$ blau, ..., $i_s$ gelb. 

Auf wie viele Arten kann man die Kugeln anordnen? $(=k)$

$n! = k\cdot i_1! \cdot i_2 \cdot ... \cdot i_s!$

$k= \frac{n!}{i_1! \cdot i_2 \cdot ... \cdot i_s!}$ ist der Multinomialkoeffizient. 

\subsection{Beispiel}
Experiment mit 3 Ausgängen $A_1, A_2, A_3$ und den dazugehörigen Wahrscheinlichkeiten $\frac{1}{2}, \frac{1}{4}, \frac{1}{4}$. Es wird $n=7$ mal wiederholt. 
$P(X_1=4, X_2=2, X_3=1)$
$=\rbr{\frac{1}{2}}^4 \cdot \rbr{\frac{1}{4}}^2 \cdot \rbr{\frac{1}{4}}^1 \cdot \frac{7!}{4!\cdot 2! \cdot 1!}$
$=\rbr{\frac{1}{2}}^4 \cdot \rbr{\frac{1}{4}}^3 \cdot \frac{1\cdot 2\cdot 3\cdot 4\cdot 5\cdot 6\cdot 7}{1\cdot 2\cdot 3\cdot 4\cdot 1\cdot 2\cdot 1}$
$=\frac{1}{16} \cdot \frac{1}{64} \cdot 105$
$=0.102$
$\approx 10 \%$