% Vorlesung vom 22.12.2015
\renewcommand{\ldate}{2015-12-22}

\subsection{Aufgabe 22.2}
Urne mit einer roten und einer schwarzen Kugel. Wir ziehen eine Kugel. Wenn wir die rote ziehen, sind wir fertig. Wenn wir die schwarze ziehen, legen wir \textbf{zwei} schwarze zurück. 
Die Zufallsgröße X entspricht der Anzahl der Ziehungen. 
$\Omega = \cbr{(r), (s,r), (s,s,r), (\underbrace{\overset{2}{s},\overset{3}{s},\overset{4}{s},\overset{5}{r}}_{X=4}), ...}$

\begin{enumerate}
\item $P(X=1) = \frac{1}{2}, P(X=4) = \frac{1}{2} \cdot \frac{2}{3} \cdot \frac{3}{4} \cdot \frac{1}{5}
\Rightarrow P(X=k) = \frac{(k-1)!}{(k+1)!} = \frac{1}{k(k+1)}$
\item $ \sum_{k=1}^{\infty} P(X=k) = \sum_{k=1}^{\infty} \frac{1}{k(k+1)}$. Mit Partialbruchzerlegung: 
$ 
\frac{1}{k(k+1)}
= \frac{A}{k} + \frac{B}{k+1}
= \frac{A(k+1) + Bk}{k(k+1)}
= \frac{A + k(A+B)}{k(k+1)}
\Rightarrow A = 1, A + B = 0 \Rightarrow B = -1
\Rightarrow \frac{1}{k} - \frac{1}{k+1}
\rightarrow = \sum_{k=1}^{\infty} \frac{1}{k} - \frac{1}{k+1}
= \rbr{\frac{1}{1} - \frac{1}{2}} + \rbr{\frac{1}{2} - \frac{1}{3}} +  \rbr{\frac{1}{3} - \frac{1}{4}} + ... \rightarrow 1 \checkmark
$
\item ges.: 
$ 
EX = \sum_{k=1}^{\infty} k \cdot P(X=k) 
= \sum_{k=1}^{\infty} k \cdot \frac{1}{k(k+1)}
= \sum_{k=1}^{\infty} \frac{1}{k+1}
= \sum_{k=2}^{\infty} \frac{1}{k}
= \frac{1}{2} + \frac{1}{3} + \frac{1}{4} + \frac{1}{5} + ...
= \infty
\Rightarrow $ Der Erwartungswert ist unendlich bzw. existiert nicht. 
\end{enumerate}

\section{Wartezeitprobleme}
Wir haben ein Treffer-Niete-Experiment. Treffer ist 1 und Niete 0. Die Wahrscheinlichkeiten sind p und q, wobei $q=1-p$. Das Experiment wird solange wiederholt, bis zum ersten mal ein Treffer kommt. 
$\Omega = \cbr{\underbrace{1}_{\omega_1}, \underbrace{01}_{\omega_2}, 001, 0001, ...}$

$p_1(\omega_j) = (1-p)^{j-1} \cdot p$
\textbf{Test:} $ \sum_{j=1}^{\infty} p_1 (\omega_j) 
= p\cdot \sum_{j=1}^{\infty} (1-p)^{j-1}
= p\cdot \sum_{j=0}^{\infty} (1-p)^{j}
\underbrace{=}_{\textrm{geom. R.}} p\cdot \frac{1}{1-(1-p)} = 1
$

$ X(\omega_j) = j-1 $ Anzahl der Nieten bis zum ersten Treffer. 

Verteilung von X: 
$
P(X=k) = 
\underbrace{(1-p)^k}_{\textrm{Nieten}} \cdot \underbrace{p}_{\textrm{Treffer}}, k=0,1,2,3,...
$. 
X heißt geometrisch verteilt.

\begin{satz}
Es sei X geometrisch verteilt $P(X=k) = (1-p)^k \cdot p$. Es gilt:

\begin{enumerate}
\item $EX = \frac{1-p}{p}$
\item $Var(X) = \frac{1-p}{p^2}$
\end{enumerate}
\end{satz}

\paragraph{Wir berechnen}
\begin{enumerate}
\item $ EX = \sum_{k=0}^{\infty} k \cdot p$
Mit 
$
\sum_{k=1}^{\infty} k \cdot x^{k-1} = \frac{1}{(1-x)^2} \Rightarrow 
= \sum_{k=1}^{\infty} k\cdot p (1-p)^{k-1} \cdot (1-p)
= p(1-p) \sum_{k=1}^{\infty} k(\underbrace{1-p}_{X})^{k-1}
= p(1-p) \cdot \frac{1}{(1-(1-p))^2}
= \frac{1-p}{p} \checkmark
$
\item $ V(X) = E(X^2) - (EX)^2 
= \underbrace{E[X\cdot (X-1)]}_{\textrm{den rechnen wir aus}} + EX - (EX)^2$\\
$
E[X\cdot (X-1)] 
= \sum_{k=0}^{\infty} k(k-1)(1-p)^k p
= p(1-p)^2 \sum_{k=2}^{\infty} k(k-1)(\underbrace{1-p}_{X})^{k-2}
$
Mit $\sum_{k=2}^{\infty} k(k-1) x^{k-2} = \frac{2}{(1-x)^3} \Rightarrow
= p(1-p)^2 \cdot \frac{2}{(1-(1-p))^3}
= \frac{2p(1-p)^2}{p^3}
= \frac{2(1-p)^2}{p^2}
\Rightarrow V(X) = \frac{2(1-p)^2}{p^2} + \frac{1-p}{p} - \frac{(1-p)^2}{p^2}
= \frac{(1-p)^2}{p^2} + \frac{1-p}{p}
= \frac{(1-p)^2 + (1-p)p}{p^2}
= \frac{(1-p) (1-p + p)}{p^2}
= \frac{1-p}{p^2}
$
\end{enumerate}