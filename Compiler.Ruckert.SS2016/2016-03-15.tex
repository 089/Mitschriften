\section{Phasen eines Compilers}

\subsection{Lexical Analysis}
Input zerlegen in kleinste sinntragende Einheiten (Tokens)\index{Token}. 

\textbf{Beispiel}\\
\begin{tabular}{|c|c|c|c|c|c|c|c|c|c|}
% \hline  &  &  &  & & & & & & \\ 
\hline if & ( & count & + & 5 & == & 20& return & 0 & ; \\ 
\hline Keyword &  & Identifier & Operator & Zahl & Vergleichsoperator & Zahl & Keyword & & Semicolon\\ 
\hline 
\end{tabular} 

\paragraph{Besonderheiten}

\begin{itemize}
\item lift = 1; lift ist ein Identifier, nicht l .. if .. t
\item if(x != 3), != ist not equal
\item if(x = ! 3) = assignment, ! not
\end{itemize}

Lexical Analysis reduziert normalerweise Daten um den Faktor 2 bis 20. 

\subsection{Syntactic Analysis}
Erzeugung eines parse tree\index{parse tree} aus dem Strom von Tokens. 

\paragraph{Beispiel}
--- Abb. 1

-> Papier gehts weiter