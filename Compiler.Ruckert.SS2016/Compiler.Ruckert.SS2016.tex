% GRUNDEINSTELLUNGEN -------------------------------------------

\documentclass[a4paper]{article}
\usepackage[ngerman]{babel}
\usepackage[utf8]{inputenc}


% EINGEBUNDENE PAKETE ------------------------------------------

\usepackage{pdfpages} % andere PDFs einbinden
\usepackage{verbatim} % program listings
\usepackage{latexsym} % Symbole zB Box
\usepackage{amssymb} % z.B. mathbb{}
\usepackage{amsmath} 
\usepackage{amsthm} % Für Lemmata, Sätze(Theoreme), Beweise (siehe unten)
\usepackage{tikz}	% Für Diagramme aus dem Programm Dia
\usepackage{makeidx} % fuer Stichwortverzeichnis
\usepackage{csquotes}
\usepackage{color}

% Verweise anklickbar machen
\usepackage{hyperref} % 1. 
\usepackage[figure]{hypcap} % 2. 


% EIGENE BEFEHLE -----------------------------------------------
% Bildbreite: Originalgröße oder falls größer als Seitenbreite/Spalte ==> skalieren
\makeatletter
\def\ScaleIfNeeded{%
\ifdim\Gin@nat@width>\linewidth
\linewidth
\else
\Gin@nat@width
\fi
}
\makeatother

% Neuer Befehl der Bilder standartisiert einbindet:
% includegraphicsdeluxe benötigt \ScaleIfNeeded und \ldate
% 1. innerhalb der figure-Umgebung mit dem Versuch, das Bild mittels [!htb] an genau dieser Position einzufügen
% 2. mit der Original- oder einer skalierten (mittels ScaleIfNeeded s.o.) Größe
% 3. mit einer Beschriftung (\caption{})
% 4. mit einem Label (\label{})
\newcommand{\includegraphicsdeluxe}[4]{
	\begin{figure}[!htb] 
	\centering
	\includegraphics[width=1\ScaleIfNeeded]{pics/\ldate_#1}
	\caption[#2]{#3}
	\label{#4}
	\end{figure}
}


% AND und OR Symbole
\newcommand{\und}{\wedge}
\newcommand{\oder}{\vee}

% Grundmengen
\newcommand{\N}{\mathbb{N}}
\newcommand{\Z}{\mathbb{Z}}
\newcommand{\Q}{\mathbb{Q}}
\newcommand{\R}{\mathbb{R}}
\newcommand{\C}{\mathbb{C}}

% größenangepasste Betragsstriche, Klammern, ...
\newcommand{\abs}[1]{\ensuremath{ \left\vert #1 \right\vert }}
\newcommand{\rbr}[1]{\ensuremath{ \left( #1 \right) }} % round brackets ()
\newcommand{\cbr}[1]{\ensuremath{ \left\lbrace #1 \right\rbrace }} % curly brackets {}
\newcommand{\sbr}[1]{\ensuremath{ \left[ #1 \right] }} % square brackets []

% Vektoren mit Klammern und so 
%\newcommand{\vektor}[1]{\ensuremath{ \rbr{ \begin{array}{c} #1 \end{array} } }} 
\newcommand{\vektor}[1]{\ensuremath{ \begin{pmatrix} #1 \end{pmatrix} }} 

% Sonderzeichen
\newcommand{\grad}{$^\circ$}
\newcommand{\gradi}{^\circ}

% Deltas
\newcommand{\Dt}{\Delta t}
\newcommand{\Dx}{\Delta x}
\newcommand{\Dy}{\Delta y}
\newcommand{\DA}{\Delta \alpha}
\newcommand{\DPH}{\Delta \varphi}
\newcommand{\Ds}{\Delta s}

\newcommand{\dt}{\delta t}
\newcommand{\dx}{\delta x}
\newcommand{\dy}{\delta y}
\newcommand{\df}{\delta f}
\newcommand{\dz}{\delta z}


% lokale Platzhalter
\newcommand{\locpl}{}

% Datum der einzelnen Lektionen definieren
\newcommand{\ldate}{2016-03-15}	% define lessiondate

% Anmerkung Professor
\newcommand{\profnote}[1]{\marginpar{\tiny{\textquote{#1}}}}

% Dokumenteneigenschaften
\newcommand{\pTitle}{Integraltransformationen}
\newcommand{\pShortName}{M. Zell}
\newcommand{\pSemester}{SS 2016}
\newcommand{\pProfessor}{Prof. Dr. Martin Leitner}

% EINRÜCKUNG ---------------------------------------------------
\setlength{\parindent}{0pt} % Erste Zeile in Absätzen nicht einrücken.

% KOPF- UND FUSSZEILE ------------------------------------------
\usepackage{fancyhdr}
\pagestyle{fancy}
\fancyhf{}
 
%Kopfzeile mittig mit Kaptilname
\fancyhead[L]{\textsf{\nouppercase{\leftmark}}}
\fancyhead[R]{\textsf{\thepage}}
%Linie oben
\renewcommand{\headrulewidth}{0.5pt}
%Fußzeile 
\fancyfoot[L]{\tiny{\pTitle}}
\fancyfoot[R]{\tiny{\pShortName, \pSemester, \ldate}}
%Linie unten
\renewcommand{\footrulewidth}{0.5pt}
 
% START DES EIGENTLICHEN DOKUMENTS -----------------------------
\author{\pShortName}
\title{Mitschrift\\\pTitle, \pSemester\\\pProfessor}

% Stichwortverzeichnis erstellen
\makeindex

\begin{document}
\maketitle
\newpage
\tableofcontents
\newpage
\listoffigures

% Sätze usw. aus asmthm
\newtheorem{satz}{Satz}[section] 
\newtheorem{defi}{Definition}[section] 
\newtheorem{lem}{Lemma}[section] 
\newtheorem{beh}{Behauptung}[section] 

% einzelne Kapitel
% \renewcommand{\ldate}{2015-10-01}

\section{Hinweise}
Diese Mitschrift basiert auf der Vorlesung \textquote{\pTitle} von \pProfessor \ im \pSemester. Du kannst sie gerne benutzen, kopieren und an andere weitergeben. Auch in der Prüfung - soweit zugelassen \footnote{\url{http://www.cs.hm.edu/meinstudium/studierenden_services/fi_pruefungskatalog.de.html}} - kannst du sie gerne als Hilfsmittel verwenden, wenn das meine Nutzung als Prüfungshilfsmittel nicht in irgendeiner Weise beeinträchtigt.\\

Natürlich besteht kein Anspruch auf Aktualität, Richtigkeit, Fortsetzung meines Angebots oder dergleichen. Sollten dir Fehler auffallen oder solltest du Verbesserungsvorschläge haben, würde ich mich über eine E-Mail (zell@hm.edu) freuen. Wenn du mir als kleines Dankeschön z.B. ein Club-Mate\footnote{\url{http://www.clubmate.de/ueber-club-mate.html}} ausgeben möchtest, findest du mich meistens hier: \url{http://fi.cs.hm.edu/fi/rest/public/timetable/group/if3b}. Wenn nicht, ist es auch ok ;-)\\

Nach der Prüfung werde ich den \LaTeX-Quelltext veröffentlichen, damit die Mitschrift weitergeführt, korrigiert und ergänzt werden kann.\\

Viele Grüße\\
\pShortName
\section{Phasen eines Compilers}

\subsection{Lexical Analysis}
Input zerlegen in kleinste sinntragende Einheiten (Tokens)\index{Token}. 

\textbf{Beispiel}\\
\begin{tabular}{|c|c|c|c|c|c|c|c|c|c|}
% \hline  &  &  &  & & & & & & \\ 
\hline if & ( & count & + & 5 & == & 20& return & 0 & ; \\ 
\hline Keyword &  & Identifier & Operator & Zahl & Vergleichsoperator & Zahl & Keyword & & Semicolon\\ 
\hline 
\end{tabular} 

\paragraph{Besonderheiten}

\begin{itemize}
\item lift = 1; lift ist ein Identifier, nicht l .. if .. t
\item if(x != 3), != ist not equal
\item if(x = ! 3) = assignment, ! not
\end{itemize}

Lexical Analysis reduziert normalerweise Daten um den Faktor 2 bis 20. 

\subsection{Syntactic Analysis}
Erzeugung eines parse tree\index{parse tree} aus dem Strom von Tokens. 

\paragraph{Beispiel}
--- Abb. 1

-> Papier gehts weiter



% Stichwortverzeichnis
% \newpage
% \renewcommand{\indexname}{Stichwortverzeichnis} % Index soll Stichwortverzeichnis heissen
% \addcontentsline{toc}{section}{Stichwortverzeichnis} % Stichwortverzeichnis soll im Inhaltsverzeichnis auftauchen
% \printindex % Stichwortverzeichnis endgueltig anzeigen
\end{document}