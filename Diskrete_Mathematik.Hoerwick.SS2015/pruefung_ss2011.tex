 % Vorlesung vom 11.06.2015
\renewcommand{\ldate}{2015-06-11}	% define lessiondate

% Angabe einbinden
% \includepdf[pages={1,3-5}]{Dateiname}

\section{Lösung zur Prüfung SS 2011}

\subsection{Aufgabe 1}

\subsubsection{1a} 
$\overbrace{x^2}^{a^2} + \overbrace{10x}^{2b} + 1 = 11 $ \marginpar{$(a+b)^2=a^2+2ab+b^2$}\\
$x^2 + 2 \cdot  5x + 5^2 = 11-1 + 5^2$\\
$(x+5)^2=10+25=35=9$\\
$x+5= \pm 3$\\
$x_1 + 5 = 3, x_1=-2=11$\\
$x_2+5=-3, x_2=-8=5$

\subsubsection{1b}
$I: x+2y=12$\\
$II: 3x+y=11 \Rightarrow y-6y=11-36=-25$\\
$-5y=-25 \Rightarrow y=5$\\
\textbf{in I einsetzen:} $x+2\cdot 5=12 \Rightarrow x=2$

\subsection{Aufgabe 2}

\subsubsection{2a}
$\mathbb{Z}_{14}^* =\{1,3,5,9,11,13\}$ \marginpar{Hinweis: In jeder Zeile/Spalte kommt jede Zahl \textbf{genau einmal} vor!}

\begin{tabular}{|c|c|c|c|c|c|c|}
\hline $\cdot$ & 1 & 3 & 5 & 9 & 11 & 13 \\ 
\hline 1 & 1 & 3 & 5 & 9 & 11 & 13 \\ 
\hline 3 & 3 & 9 & 1 & 13 & 5 & 11 \\ 
\hline 5 & 5 & 1 & 11 & 3 & 13 & 9 \\ 
\hline 9 & 9 & 13 & 3 & 11 & 1 & 5 \\ 
\hline 11 & 11 & 5 & 13 & 1 & 9 & 3 \\ 
\hline 13 & 13 & 11 & 9 & 5 & 3 & 1 \\ 
\hline 
\end{tabular} 

\subsubsection{2b}
$\pi_1(a) \cdot  \pi_2(b) \cdot  \pi_3(c) = 1$ \marginpar{(3,11,$\cdot$)}\\
$\pi_1(3) \cdot  \pi_2(11) \cdot  \pi_3(x) = 1$
$11 \cdot  3 \cdot  \pi_3(x) =1$\\
$5 \cdot  \pi_3(x)=1$ \marginpar{Löse: 5 mal was ist 1?}\\
$\pi_3(x) =3 \Rightarrow x=9$ 

\subsection{Aufgabe 3}
$1,2,3,5,8,13,...$\\
$fib(n)=fib(n-1)+fib(n-2)$\\
Anzahl der Möglichkeiten: $g(n)$\\

\paragraph{Induktionsanfang:} 

% Nr. 1
\includegraphicsdeluxe{ss2011-3_induktionsanfang.jpg}{Der Induktionsanfang für $n=1, n=2$}{fig:ss2011-3_induktionsanfang}

$n=1$\\
$g(1)=1, fib(1)=1$\\

$n=2$\\
$g(2)=2, fib(2)=2$\\

\paragraph{Induktionsschritt:} Anfänge (Abb. \ref{fig:ss2011-3_induktionsschritt}):

\includegraphicsdeluxe{ss2011-3_induktionsschritt.jpg}{Mögliche Anfänge bei den Dominosteinen}{fig:ss2011-3_induktionsschritt}

$g(n)=g(n-1)+g(n-2)=fib(n-1)+fib(n-2)=fib(n)$

\subsection{Aufgabe 4}

\subsubsection{Direktlösung}
A hat 33 Zähne (0 bis 32), B hat 14 Lücken (0 bis 13). 
\paragraph{Wann greift Zahl 6 von A in die Lücke 10 von B?} x ist die Anzahl der Zähne. \\
$z+k\cdot 33=9+t\cdot 14=x$\\
$x \equiv 2 \textrm{ mod } 33$ (chin. Restsatz)\\
$x \equiv 9 \textrm{ mod } 14$\\

\paragraph{In welche Lücken greift der Zahn 6 von A?} \marginpar{(mod 14!)} 
$3, 3-33=-30=12, 12-33=-21=7, 7-33=-26=2, 2-33=-31=11, 11-33=-22=6, 6-33=-27=1, 1-33=-32=\underline{\textbf{10}}$\\
$x=2+7\cdot 33=\underline{\textbf{233}}=9+t\cdot 14$\\
$t=\frac{233-9}{14}=16$

\subsubsection{Lösung mit chinesischem Restsatz}
$x \equiv a_1 (=2) \textrm{ mod } m_1 (=33)$\\
$x \equiv a_2 (=9) \textrm{ mod } m_2 (=14)$\\
$m_1 \cdot  m_2 = 462$\\
$M_1=14, M_2=33$\\
$y_i \cdot  M_i \equiv 1 \textrm{ mod } m_i$ \marginpar{$i=\frac{1}{2}$}\\
$y_1 \cdot  14 \equiv 1 \textrm{ mod } 33$

\paragraph{euklidischer Algorithmus} $ggT(33,14)$\\
$33=2\cdot 14+5$\\
$14=2\cdot 5+4$\\
$5=1\cdot 4+1$

\paragraph{Erweiterter euklidischer Algorithmus} Die bisherigen Werte kannten wir schon, daher brauchen wir den erweiterten euklidischen Algorithmus. \\
$1=5-4=5-(14-25)=3\cdot 5-14=3(33-2\cdot 14)-14=3\cdot 33-7\cdot 14=1 \Rightarrow (-7)\cdot 14 \equiv 1 \textrm{ mod } 33$\\
$y_1=-7=26=y_1$\\
$y_2 \cdot  M_2 \equiv 1 \textrm{ mod } m_2$\\
$y_2 \cdot  33 \equiv 1 \textrm{ mod } 14$\\
$3 \cdot  33 \equiv 1 \textrm{ mod } 14$\\
$y_2 = 3$\\
$x=\sum_{i=1}^{2} a_i y_i M_i = 2 \cdot  26 \cdot  14 + 9 \cdot  3 \cdot  33 = 1619$
x ist eindeutig modulo $m_1 \cdot  m_2 = 462$\\
Gesucht ist das \textbf{erste Greifen} des Zahnes: $1619 \pm k \cdot  462, x=233$ (mod 462)
