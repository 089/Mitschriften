 % Vorlesung vom 22.06.2015
\renewcommand{\ldate}{2015-06-22}	% define lessiondate

\section{Einzelne Aufgaben}

\subsection{Dominosteine in 3xn-Feld unterbringen}
Wie viele Möglichkeiten a(n) gibt es 1x2-Steine anzuordnen? Wenn n ungerade ist, geht es nicht. 
\includegraphicsdeluxe{dominofelddarstellung.jpg}{Wie kann man die 1x2 Steine im Feld unterbringen?}{fig:dominofelddarstellung} % Nr. 1
Wenn man die Anfänge in Abb. \ref{fig:dominofelddarstellung} betrachtet kommt man auf folgende Formeln für die Möglichkeiten: 

\begin{itemize}
\item $a(n)=a(n-2)+2 b(n)$
\item $b(n)=a(n-2)+b (n-2)$
\end{itemize}

\begin{tabular}{|c|c|c|c|c|}
\hline n & 2 & 4 & 6 & 8 \\ 
\hline a(n) & 3 & $3+2\cdot 4=11$ & $11+2\cdot 15=41$ & $41+2\cdot 56=153$ \\ 
\hline b(n) & 1 & $3+1=4$ & $11+4=15$ & $41+15=56$ \\ 
\hline 
\end{tabular} 

\subsection{Rechnen im $\Z_{11}$}
Wir rechnen in $(\Z_{11},+,\cdot)$. 11 ist eine Primzahl also ist das ein Körper. 

\subsubsection{Lineares Gleichungssystem}
I: $x+3y=8$\\
II: $2x+y=4$\\

II-2I: $y-6y=4-16$\\
$-5y=-12$\\
$5y=12=23=34=45$\\
$\Rightarrow y=9$\\

in I: $x+3\cdot 9=8$\\
$x=8-27=-19$\\
$x=3$

\subsubsection{Quadratische Gleichung}
$x^2+4x=10$ mit $(a+b)^2 = a^2+2ab+b^2 \Rightarrow b=2$\\
$x^2+4x+w^2=10+2^2$\\
$(x+2)^2=14=3$\\
Die Lösungen finden wir mittels Ausprobieren: 
$2^2=4, 3^2=9, 4^2=16=5,$ \textbf{$5^2=25=3$}\\
$x+2=\pm 5$\\
$x_1 + 2 = 5 \Rightarrow x_1=3$\\
$x_2+2=-5 \Rightarrow x_2=-7=4$\\

\subsection{Kombinatorikaufgabe}
10 Ehepaare sitzen an einem langen Tisch. Auf einer Seite die Männer, auf der anderen die Frauen. Wie groß ist die Wahrscheinlichkeit, dass sich kein Ehepaar gegenüber sitzt? \\
Das klingt nach Permutationen: \\
$
\begin{array}{ccccccc}
\textrm{Männer} & 1 & 2 & 3 & 4 & ... & 10 \\ 
 & \downarrow & \downarrow & \downarrow & \downarrow & \downarrow & \downarrow \\ 
\textrm{Frauen} & 3 & 4 & 7 & 10 & ... & 5
\end{array} 
$\\
Damit kann man die Aufgabe neu formulieren: Wie groß ist die Wahrscheinlichkeit, dass die Permutationen keinen Fixpunkt hat?\\
alle Permutationen: n!\\
Permutationen ohne Fixpunkt: $a(n)=n!\underbrace{(1-\frac{1}{1!}+\frac{1}{2!}-\frac{1}{3!}+...+(-1)^n \frac{1}{n!})}_{\approx e^{-1}} \approx n! e^{-1}= \frac{10!}{e}$\\
P(kein Fixpunkt)$=\frac{\textrm{günstige Fälle}}{\textrm{alle Fälle}}=\frac{\frac{n!}{e}}{n!}=\frac{1}{e}=0,367$

\subsection{Einheitengruppe}
$(\Z_{16}^* ,\cdot)$ Einheitengruppe (teilerfremd zu 16): $\Z_{16}^*  = \{1,3,5,7,9,11,13,15\}$\\
Jetzt machen wir einen Code c über $\Z_{16}^* $.\\
$c=\{(a,b,c) : a b c = 1, a,b,c \in \Z_{16}^* \}$\\
\paragraph{Ergänze $(5,11,\cdot)$} zu einem Codewort.\\
$5\cdot 11\cdot x=1$\\
$55 x = 1$\\
$7 x = 1$ \\
Jetzt probieren wir die Werte aus der o.g. Einheitengruppe aus:\\
$7\cdot 3=21=5$\\
$7\cdot 3=35=3$\\
$7\cdot 7=49=1$\\
$\Rightarrow x=7$

\paragraph{Aus wie vielen Elementen besteht der Code?}
$(a,b,\cdot)$ mit a,b beliebig wählen und $\cdot$ rechnen wir aus.
Für a und b gibt es jeweils 8 Möglichkeiten (Anzahl der Elemente der Einheitengruppe), also 64 Elemente. Der Code hat also 64 Codewörter. 

\subsection{Verschlüsseln und Entschlüsseln mit Permutationen}
\paragraph{Wir haben eine gegebene Permutation}
$\pi: 
\begin{array}{ccccccc}
1 & 2 & 3 & 4 & 5 & 6 & 7 \\ 
\downarrow & \downarrow & \downarrow & \downarrow & \downarrow & \downarrow & \downarrow \\ 
3 & 7 & 1 & 5 & 4 & 2 & 6
\end{array} 
$
\paragraph{Suche die inverse Permutation}
$\pi^{-1}: 
\begin{array}{ccccccc}
1 & 2 & 3 & 4 & 5 & 6 & 7 \\ 
\downarrow & \downarrow & \downarrow & \downarrow & \downarrow & \downarrow & \downarrow \\ 
3 & 6 & 1 & 5 & 4 & 7 & 2
\end{array} 
$

\paragraph{Damit kann man Wörter der Länge 7 verschlüsseln.}
$c(i):=m(\pi(i))$\\

\paragraph{Verschlüssle $m=(C,A,B,C,D,E,A)$}
$\Rightarrow c=(B,A,C,D,C,A,E)$

\paragraph{Entschlüssle c (mit $\pi^{-1}$)}
$m(i) := c(\pi^{-1}(i))$\\
$m=(C,A,B,C,D,E,A) \checkmark$

\subsection{Graphentheorie}
Hat der gegebene Graph (Abb. \ref{fig:aufgaben_graphen6}) einen eulerschen Kreis oder eine eulersche Linie? 
\includegraphicsdeluxe{aufgaben_graphen6.jpg}{Eulersche Linie und eulerscher Kreis}{fig:aufgaben_graphen6} % Nr. 2

\subsubsection{eulersche Linie}
Es gibt genau zwei Knoten (A, E) mit ungeradem Grad $\Rightarrow$ eulersche Linie (rot). Beim Einzeichnen muss man bei einem Knoten mit ungeradem Grad (z.B. A oder E) starten! 

\subsubsection{eulerscher Kreis}
Füge eine zusätzliche Kante ein, damit ein eulerscher Kreis (nur gerade Grade) entsteht (blau). 
