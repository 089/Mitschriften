 % Vorlesung vom 15.06.2015
\renewcommand{\ldate}{2015-06-15}	% define lessiondate

% Angabe einbinden
% \includepdf[pages={1,3-5}]{Dateiname}

\section{Lösung zur Prüfung SS 2012}

\subsection{Aufgabe 1}

\subsubsection{a}
	sparen wir uns
\subsubsection{b}
	sparen wir uns
	
\subsection{Aufgabe 2}
$(\Z_{13},+,\cdot)$ Körper

\subsubsection{a}
$x+7y=5$\\
$5x+y=8$\\
\textbf{II-5 * I:} $y-35y=8-25$\\
$-34y=-17$\\
$8y=4$\\
$4\cdot 2y=4$\\
$2y=1$\\
$y=7$\\
\textbf{in I:} $x+7\cdot 7=5$\\
$x=-44=8$

\subsubsection{b}
$\underbrace{x^2}_{a^2}+\underbrace{3x}_{2b}=2$ mit $(a+b)^2=a^2+2ab+b^2; 2b=3 \Rightarrow 2\cdot 8=16=3 \Rightarrow b=8$\\
$x^2+2\cdot 8x+8^2=2+8^2$\\
$(x+8)^2=2+64=66$\\
$(x+8)^2=1$\\
$x_1+8=1 \Rightarrow x_1=-7=6$\\
$x_2+8=-1 \Rightarrow x_2=-9=4$
\textbf{Test:} $x=4: 16+12=2 \checkmark$\\
$x=6: 36+18=2 \checkmark$

\subsection{Aufgabe 3}

\subsubsection{a} $\Z_{10}^* =\{1,3,7,9\}$:
\begin{tabular}{|c|c|c|c|c|}
\hline $\cdot$ & 1 & 3 & 7 & 9 \\ 
\hline 1 & 1 & 3 & 7 & 9 \\ 
\hline 3 & 3 & 9 & 1 & 7 \\ 
\hline 7 & 7 & 1 & 9 & 3 \\ 
\hline 9 & 9 & 7 & 3 & 1 \\
\hline 
\end{tabular} 
\\
\subsubsection{b} $c=\{(a,b,c):a,b,c \in \Z_{10}^* $ und $\pi_1(a) \cdot  \pi_2(b) \cdot  \pi_3(c)=1\}$\\
Ergänze: $(3,9,x)$\\
$\pi_1(3) \cdot  \pi_2(9) \cdot  \pi_3(x)=1$\\
$7\cdot 7\cdot \pi_3(x)=1$\\
$9\cdot \pi_3(x)=1$\\
$\pi_3(x)=9 \Rightarrow x=3$

\subsection{Aufgabe 4}

\subsubsection{a} Die Gesamtzahl entspricht der Anzahl der Permutationen mit $n=20$, also $20!$ Möglichkeiten. 

\subsubsection{b} Richtig ankommen sind die Fixpunkte, d.h. Anzahl der Permutationen ohne Fixpunkt (keiner kommt richtig an).\\
$a(n)=n!\underbrace{(1-\frac{1}{1!}+\frac{1}{2!}-\frac{1}{3!}+...+(-1)^n \frac{1}{n!})}_{\approx e^{-1}} \approx n! e^{-1}= \frac{20!}{e}$

\subsubsection{c}$P=\frac{\textrm{günstige Fälle}}{\textrm{alle Fälle}}=\frac{20!-\frac{20!}{e}}{20!} = 1-\frac{1}{e}=0,63$

\subsection{Aufgabe 5}
Zahn 10 von A in Lücke 8 von B?\\
A: $41+t\cdot 45=x \Rightarrow x\equiv 41$ mod $45$ \marginpar{simultane Kongruenz chin. Restsatz}\\
B: $33+s\cdot 38=x \Rightarrow x\equiv 33$ mod $38$\\
45 ($3\cdot 3\cdot 5$) und 38 ($2\cdot 19$) sind teilerfremd.

\textbf{chinesischer Restsatz}\\
$x\equiv a_1$ mod $m_1$ mit $a_1=41,m_1=45$\\
$x\equiv a_2$ mod $m_2$ mit $a_2=33, m_2=38$\\
$\Rightarrow m=m_1\cdot m_2=1710$\\
$M_1=m_2=38, M_2=45$\\
$y_1\cdot M_1\equiv 1$ mod $m_1 \Leftrightarrow y_1\cdot 38\equiv 1$ mod $45$\\
$y_2\cdot M_2\equiv 1$ mod $m_2 \Leftrightarrow y_2\cdot 45\equiv 1$ mod $38$

\textbf{Zwischenrechnung: mit euklidischem Algorithmus} 
$ggT(45,38): 45=1\cdot 38+7 \Rightarrow 38=5\cdot 7+3 \Rightarrow 7=2\cdot 3+1$\\
$1=7-2\cdot 3=7-2(38-5\cdot 7)=11\cdot 7-2\cdot 38=11(45-38)-2\cdot 38=11\cdot 45-13\cdot 38=1$

\textbf{Fortsetzung}
$(-13)\cdot 38\equiv1$ mod $45 \Rightarrow y_1=-13=32$\\
$11\cdot 45\equiv 1$ mod $38 \Rightarrow y_2=11$

\textbf{x ausrechnen} $x=a_1\cdot y_1\cdot M_1 + a_2\cdot y_2\cdot M_2=41\cdot 32\cdot 38+33\cdot 11\cdot 45=66191$. Diese Lösung ist eindeutig mod $m=1710$. Das erste mal trifft der Zahn 10 also in die Lücke 8 (x so klein wie möglich):\\
$x=1211$
