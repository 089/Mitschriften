% Vorlesung vom 11.05.2015
\renewcommand{\ldate}{2015-05-11}	% define lessiondate

\section{Fehlererkennung}

\subsection{Fehlererkennender Code}
Teilmenge C von V. Sender schickt $ c \in C$. Empfänger empfängt c' aus V. Ist c' in C akzeptiert er, sonst lehnt er ab.
\paragraph{Beispiel (Übermitteln einer Vierstellige Zahl ):} Man fügt eine fünfte Zahl (Kontrollzahl) hinzu, sodass die $ Quersumme = 0 \ \mbox{mod} \ 10 $ ist. \marginpar{\textit{Es soll ein Vielfaches von 10 rauskommen}}\\
$ V=\{(a_1,...,a_5) : a_i = 0,1,...,9\} $\\
$ C=\{(a_1,...,a_5) \in V: a_1 + a_2 + a_3 + a_4 + a_5 = 0 \ \mbox{mod} \ 10\} $\\
\textbf{Code n zur Basis q:} $ C=\{(a_1,...,a_5) : 0 \leq a_i \leq q-1\} $\\
\textbf{Paritätscode:} Code der Länge n zur Basis q und 
\[ C \equiv \{(a_1,...,a_5) : 0 \le a_1 \le q-1 \ \mbox{und} \ a_1 + ... + a_n = 0 \equiv q \}\]

\paragraph{Beispiel:} $n=5, q=11$. Ist der Code $ (3,0,10,4,5) \in C$?\\
$ 3+0+10+4+5=22 $

\paragraph{Satz} 
Jeder Paritätscode erkennt Einzelfehler (eine Stelle ändern). 

\paragraph{Beweis}
Das Codewort ist $a_1,...,a_i,...,a_n $ Codewort\\
Es kommt zu einem Fehler $a_1,...,\underbrace{a_i'}_{Fehler},...,a_n \ $\\
\[a_1 + a_i' + ... + a_n - \underbrace{(a_1 + ... + a_i + ... + a_n)}_{0} = 
a_i' - a_i \neq 0 \ \mbox{mod} \ q\]\\
\textbf{Vertauschungsfehler:} $a_i$ und $a_j$ werden vertauscht. Ein Paritätscode erkennt keinen Vertauschungsfehler!

\paragraph{Idee: Gewichte!} Basis q, n Stellen, Gewichte $g_i \ \mbox{mit} \ 1 \le g_i \le q-1$ \\
$ g_1 a_1 + ... + g_n a_n \equiv 0 \ \mbox{mod} \ q$ 

\paragraph{Satz:}
Ist $g_n$ teilerfremd zu q, so kann man immer die Kontrollziffer $a_n$ ausrechnen.

\paragraph{Beweis}
$ g_1 a_1 + ... + g_n a_n = 0$ \marginpar{mod $q$ !}\\ 
$ g_n a_n = - g_1 a_1 - ... - g_{n-1} a_{n-1}$\\
\[a_n = \underbrace{g_{n}^{-1}}_{\mbox{existiert, da } g_n \ \mbox{teilerfremd zu } q.} (- g_1 a_1 - ... - g_{n-1} a_{n-1})
\]


\paragraph{Satz:} Ein Paritätscode mit Gewichten erkennt jeden Einzelfehler der Stelle $a_i$, wenn $g_i$ und $q$ teilerfremd sind.

\paragraph{Beweis:} 
$a_1 \rightarrow a_1' (falsch)$\\
$g_1 a_1' + ... + g_n a_n = $\\
$g_1 a_1' + ... + g_n a_n - (g_1 a_1 + ... + g_n a_n) = $\\
$g_1 a_1' - g_1 a_1 = \underbrace{\underbrace{g_1}_{\neq 0} \underbrace{(a_1' - a_1)}_{\neq 0}}_{\mbox{Einheit (} g_1 \ \mbox{teilerfremd zu }q \ \mbox{). Kein Nullteiler.}} \neq 0$\\

\paragraph{Folgerung:} Damit man alle Einzelfehler erkennen kann, müssen $g_i$ teilerfremd zu $q$ sein.

\paragraph{Satz:} Ein Paritätscode mit Gewichten erkennt den Vertauschungsfehler $a_i \leftrightarrow a_j$, wenn $g_i - g_j$ teilerfremd zu q ist. 

\paragraph{Beweis:} Vertauschungsfehler, z.B.: $ a_1 \leftrightarrow a_2$\\
$g_1 a_2 + g_2 a_1 + g_3 a_3 + ... + g_n a_n =$\\
$g_1 a_2 + g_2 a_1 + ... + g_n a_n - (g_1 a_1 + ... + g_n a_n) =$\\
$g_1 a_2 + g_2 a_1 - g_1 a_1 - g_2 a_2 =$\\
$g_1(a_2 - a_1) + g_2(a_1 - a_2) =$\\
$g_1(a_2 - a_1) - g_2(a_2 - a_1) =$\\
$\underbrace{\underbrace{(g_1 - g_2)}_{\neq 0} \underbrace{(a_2-a_1)}_{\neq 0}}_{Einheit, kein Nullteiler} \neq 0$\marginpar{mod q}\\


\paragraph{Beispiel:} Ein Paritätscode der Länge 10 zur Basis $q=11$.\\ 

\begin{tabular}{ccccccccccc}
 & $a_1$ & $a_2$ & $a_3$ & $a_4$ & $a_5$ & $a_6 $ & $a_7 $ & $a_8 $ & $a_9 $ & $a_{10}$\\ 
$g_i$ & 10 & 9 & 8 & 7 & 6 & 5 & 4 & 3 & 2 & 1\\ 
\end{tabular} 
Erkennt jeden Einzelfehler und jeden Vertauschungsfehler! Das ist der ISBN-Code bei den Büchern (10=X).

\paragraph{Beispiel (ISBN-Code):} "Die letzte Stelle \textbf{2} kann nicht einfach gewählt werden, sondern muss stimmen! "\\
\begin{tabular}{c|cccccccccc}
 & $a_1$ & $a_2$ & $a_3$ & $a_4$ & $a_5$ & $a_6 $ & $a_7 $ & $a_8 $ & $a_9 $ & $a_{10}$\\ \hline
 & 3 & 1 & 5 & 7 & 2 & 2 & 1 & 8 & 9 & \textbf{2}\\
$g_i$ & 10 & 9 & 8 & 7 & 6 & 5 & 4 & 3 & 2 & 1\\ 
Produkt 30 & 9 & 40 & 49 & 12 & 10 & 4 & 24 & 18 & \\
mod 11 & 8 & 9 & 7 & 5 & 1 & 10 & 4 & 2 & 7 & \textbf{2}\\
\end{tabular} 
\marginpar{$\sum=8+9+7+5+1+10+4+2+7=53=9$} 

\subsection{Codes über Gruppen}
\paragraph{Wiederholung:}
$(G,\cdot)$ ist eine Gruppe, wenn:
\begin{itemize}
\item $\cdot$ ist assoziativ
\item Es gibt ein neutrales Element e
\item Jedes a hat ein Inverses b, d.h. $ab=ba=e$
\end{itemize}
Das Inverse ist eindeutig und heißt $a^{-1}$

\paragraph{Satz}
Gleichungen der Form $ax = b$ sind eindeutig lösbar.

\paragraph{Beweis:}
\begin{itemize}
\item Sei x eine Lösung $\Rightarrow a x = b \Rightarrow x = a^{-1} b$
\item Setze $x = a^{-1} b \Rightarrow a a^{-1} b = b$
\end{itemize}
$\Box$

\paragraph{Satz:} 
Die Abbildung 
\begin{equation*}
 a_l:\qquad	
 \left\{
 \begin{aligned}
	G \rightarrow G\\
	x \rightarrow ax
	\end{aligned}
	\right.
\end{equation*}
ist bijektiv. 

\paragraph{Beweis:}
\begin{itemize}
\item injektiv: Sei $ax=ay \Rightarrow x=y$
\item surjektiv: Falls $\vert G \vert$ endlich (klar!), $ax = b; x = a^{-1} b$
\end{itemize}
$\Box$

\paragraph{Beispiel 1:}
$(\mathbb{Z},+), (\mathbb{Z}_n,+)$, $(\mathbb{Z}_n^* ,\cdot), (\mathbb{R},+), (\mathbb{R}^* ,\cdot)$ Gruppen

\paragraph{Beispiel 2:}
Menge M: $G = \{f \colon M \rightarrow M, \mbox{mit f bijektiv}\}$ \\
Komposition: (Hintereinanderausführung) $(G,\cdot) \ \mbox{Gruppe.} \ e=id$  $id(x)=x$. Das Inverse von f ist die Umkehrabbildung $f^{-1}$

\paragraph{Beispiel 3:}
Eine bijektive Abbildung der Ebene in sich heißt \underline{Kongruenz}, wenn sie die Abstände erhält: $ d(x,y) = d(\varphi(x), \varphi(y)) \Rightarrow$ Winkel bleiben erhalten. \\
Die Kongruenzen (vgl. Abb. \ref{fig:kongruenzen}) bilden bezüglich $\circ$ eine Gruppe.
\begin{itemize}
\item Verschiebungen
\item Spiegelungen
\item Drehungen
\item Gleitspiegelungen
\end{itemize}


\includegraphicsdeluxe{kongruenzen.jpg}{Das sind alle Kongruenzen}{fig:kongruenzen}

\paragraph{Beispiel 4 (Symmetrie einer ebenen Figur):}
Alle Kongruenzen, welche die Figur auf sich abbilden, bilden eine Gruppe (vgl. Abb. \ref{fig:rechteck}). 


\includegraphicsdeluxe{rechteck.jpg}{Spiegelungen an h,g, Drehungen um Z um 180$^\circ$ und id}{fig:rechteck}


\textbf{Gruppentafel} \marginpar{\textbf{g}: wegen $dr \circ h = g$ (erst h, dann dr)\\ $1 \rightarrow 4$\\ $2 \rightarrow 3$\\ $3 \rightarrow 2$\\ $4 \rightarrow 1$}\\
\begin{tabular}{|c|c|c|c|c|}
\hline $ \circ $ & id & dr & h & g \\ 
\hline id & id & dr & h & g \\ 
\hline dr & dr & id & \textbf{g} & h \\ 
\hline h & h & g & id & dr \\ 
\hline g & g & h & dr & id \\ 
\hline 
\end{tabular}
\\

\paragraph{Definition:} 
Code über Gruppe $(G,\cdot)$ der Länge $n$ und Kontrollsymbol $c \in G$: $\{(a_1,...,a_n) \colon a_i \in G \ \mbox{und} \ a_1 a_2 ... a_n = c\}$

\paragraph{Satz:}Ein Gruppencode erkennt Einzelfehler. 

\paragraph{Beweis:}  
\[
\begin{array}{c}
(1) \ a_1 (a_2 ... a_n) = c\\
(2) \ a_1' (a_2 ... a_n) = c \\
\Rightarrow a_1 = a_1' \Box\\
\end{array}
\]

Ein Gruppencode kann einen Vertauschungsfehler erkennen oder nicht. Ist G kommutativ, so werden Vertauschungsfehler \textbf{nicht} erkannt. \marginpar{Eine Gruppe eignet sich umso besser, je weniger sie kommutativ ist.}\\
$a_1 a_2 (a_3 ... a_n) = c$\\
$a_2 a_1 (a_3 ... a_n) = c$\\
$\Rightarrow a_1 a_2 = a_2 a_1$\\

\subsection{Permutationen}
Eine Permutation ist eine bijektive Abbildung einer endlichen Menge auf sich. 
$ \pi_1:  
\begin{array}{ccccc}
1 & 2 & 3 & 4 & 5 \\ 
 \downarrow  &  \downarrow  &  \downarrow  &  \downarrow  &  \downarrow  \\ 
2 & 3 & 1 & 5 & 4
\end{array} $
\\
$ \pi_2: 
\begin{array}{ccccc}
1 & 2 & 3 & 4 & 5 \\ 
\downarrow & \downarrow & \downarrow & \downarrow & \downarrow \\ 
5 & 1 & 2 & 4 & 3
\end{array} $
\\
$ \pi_1 \circ \pi_2
\begin{array}{ccccc}
1 & 2 & 3 & 4 & 5 \\ 
\downarrow & \downarrow & \downarrow & \downarrow & \downarrow \\ 
1 & 2 & 5 & 3 & 4
\end{array} $
\\
$ \pi_2^{-1} 
\begin{array}{ccccc}
1 & 2 & 3 & 4 & 5 \\ 
\downarrow & \downarrow & \downarrow & \downarrow & \downarrow \\ 
2 & 3 & 5 & 4 & 1
\end{array} $

\paragraph{Gruppencode mit Permutationen:} 
Länge n, Gruppe G, Permutationen der Gruppe: $ \pi_1,  \pi_2, ...,  \pi_n$, Kontrollsymbol $ c \subset G$.\\
$ C = \{(a_1, ..., a_n) \colon a_i \in \ \mbox{G und} \ \pi_1(a_1) \pi_2(a_2) ....  \pi_n(a_n) = c\}$

\paragraph{Satz:}
Ein Gruppencode mit Permutationen erkennt Einzelfehler. 

\paragraph{Beweis:}
$\pi_1(a_1) ....  \pi_n(a_n) = c$\\
$\pi_1(a_1')  \pi_2(a_2) ....  \pi_n(a_n) = c$\\
$\Rightarrow \pi_1(a_1) = \pi_1(a_1') \Rightarrow a_1 = a_1'$\\
$\Box$

\paragraph{Satz:}
Ein Gruppencode mit Permutationen erkennt den Vertauschungsfehler der Stellen $i, i+1$, wenn Folgendes gilt:
\begin{itemize}
\item $\pi_i(g) \cdot  \pi_{i+1}(h) \neq \pi_i(h) \cdot  \pi_{i+1}(g) \ \forall g \neq h$
\end{itemize}
"klar" $\Box$

\paragraph{Beispiel (Code der deutschen Geldscheine):}
Gruppe: \textit{Diedergruppe} Ordnung 10. Das sind die Symmetrien des regelmäßigen 5-Ecks:

\includegraphicsdeluxe{fuenfeck.jpg}{Fünfeck: 5 Spiegelungen, 4 Drehungen, id (10)}{fig:fuenfeck}

Bilden Gruppe (10 Elemente). Wir bezeichnen die Symmetrien mit $0,1,...,9 \ (0 \widehat{=} id)$. Der Code soll die Länge n = 11 haben. Das Kontrollsymbol c sei 0. Wir benötigen 11 Permutationen. 

$ \pi_1:
\begin{array}{cccccccccc}
0 & 1 & 2 & 3 & 4 & 5 & 6 & 7 & 8 & 9 \\ 
\downarrow & \downarrow & \downarrow & \downarrow & \downarrow & \downarrow & \downarrow & \downarrow & \downarrow & \downarrow \\ 
1 & 5 & 7 & 6 & 2 & 8 & 3 & 0 & 9 & 4
\end{array} $
\\
$ \pi_2 = \pi_1 \circ \pi_1 = \pi_1^2 $ \\
$ \pi_3 = \pi_1 \circ \pi_1 \circ \pi_1 = \pi_1^3 $ \\
$ \vdots $ \\
$ \pi_{11} = \pi_1^{11} $ \\
$ Code = \{(a_1,a_2,...,a_{11}) \colon 0 \le a_i \le 9 \ \mbox{und} \ \pi_1(a_1) \pi_2(a_2) ... \pi_{11}(a_{11}) = 0 \} $ \\ 
Wie berechnet man die "Kontrollziffer" $a_{11}$?\marginpar{$c = 0$}\\
$\pi_1(a_1) ... \pi_{11}(a_{11}) = c $\\
$\pi_{11}(a_{11}) = [\pi_1(a_1) ... \pi_{10}(a_{10})]^{-1} \cdot  c$\\
$a_{11}=\pi_{11}^{-1} [(\pi_{1}(a_{1} ... \pi_{10}(a_{10}))^{-1} \cdot  c]$

\paragraph{Beispiel (Symmetrie eines gleichseitigen Dreiecks)}
Spiegelungen w1, w2, w3, Drehungen um Z 120, 240 und id (vgl. Abb. \ref{fig:gleichseitigesDreieck})

\includegraphicsdeluxe{gleichseitiges-dreieck.jpg}{Gleichseitiges Dreieck}{fig:gleichseitigesDreieck}


\textbf{Gruppentafel}\\
\begin{tabular}{|c|c|c|c|c|c|c|}
\hline $\circ$ & id & w1 & w2 & w3 & 120 & 240 \\ 
\hline id & id & w1 & w2 & w3 & 120 & 240 \\ 
\hline w1 & w1 & id & 120 & 240 & w2 & w3 \\ 
\hline w2 & w2 & 240 & id & 120 & w3 & w1 \\ 
\hline w3 & w3 & 120 & 240 & id & w1 & w2 \\ 
\hline 120 & 120 & w3 & w1 & w2 & 240 & id \\ 
\hline 240 & 240 & w2 & w3 & w1 & id & 120 \\ 
\hline 
\end{tabular} 
