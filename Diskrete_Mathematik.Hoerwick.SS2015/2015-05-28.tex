% Vorlesung vom 28.05.2015
\renewcommand{\ldate}{2015-05-28}	% define lessiondate

\section{Symmetrien des gleichseitigen Dreiecks (Wiederholung)}


% Nr. 1
\includegraphicsdeluxe{gleichseitiges_dreieck.jpg}{Spiegelungen an w1, w2, w3, Drehungen 120, 240, id}{fig:gleichseitiges_dreieck}


\begin{tabular}{|c|c|c|c|c|c|c|}
\hline $\circ$ & id & w1 & w2 & w3 & 120 & 240 \\ 
\hline id & id & w1 & w2 & w3 & 120 & 240 \\ 
\hline w1 & w1 & id & 120 & 240 & w2 & w3 \\ 
\hline w2 & w2 & 240 & id & 120 & X & w1 \\ 
\hline w3 & w3 & 120 & 240 & id & w1 & w2 \\ 
\hline 120 & 120 & w3 & w1 & w2 & 240 & id \\ 
\hline 240 & 240 & w2 & w3 & w1 & id & 120 \\ 
\hline 
\end{tabular} \\
X machen wir ausführlich: $w2 \circ 120$, $ 1 \rightarrow 2 $, $ 2 \rightarrow 1 $, $ 3 \rightarrow 3 $, $ \Rightarrow X = w3 $

\paragraph{Gruppencode (ohne Permutationen)}der Länge $n=7$ mit Kontrollsymbol $c=id$: $w1, 120, w1, w3, 120, w3, x$, berechne x passend.\\
$w1 \circ 120 \circ w1 \circ w3 \circ 120 \circ w3 \circ x = id$\\
$(w1 \circ 120) \circ (w1 \circ w3) \circ (120 \circ w3) \circ x = id$\\
$(w2 \circ 240) \circ w2 \circ x = id$ \marginpar{Klammerung wegen Assoziativität beliebig!}\\
$(w1 \circ w2) \circ x = id$\\
$120 \circ x = id$\\
$\Rightarrow x=240$

\paragraph{Gruppencode (mit Permutationen)}der Länge $n=4$ mit Permutationen, Kontrollsymbol $c=id$\\
$\pi_1$: 
$\begin{array}{cccccc}
id & w1 & w2 & w3 & 120 & 240 \\ 
\downarrow & \downarrow & \downarrow & \downarrow & \downarrow & \downarrow \\ 
w1 & 120 & id & 240 & w2 & w3
\end{array}$ 
\\
$\pi_2$: 
$\begin{array}{cccccc}
id & w1 & w2 & w3 & 120 & 240 \\ 
\downarrow & \downarrow & \downarrow & \downarrow & \downarrow & \downarrow \\ 
120 & id & 240 & w1 & w2 & w3
\end{array}$ 
\\
$\pi_3$: 
$\begin{array}{cccccc}
id & w1 & w2 & w3 & 120 & 240 \\ 
\downarrow & \downarrow & \downarrow & \downarrow & \downarrow & \downarrow \\ 
w3 & 120 & w1 & 240 & w2 & id
\end{array}$ 
\\
$\pi_4$: 
$\begin{array}{cccccc}
id & w1 & w2 & w3 & 120 & 240 \\ 
\downarrow & \downarrow & \downarrow & \downarrow & \downarrow & \downarrow \\ 
240 & w3 & w1 & w2 & id & 120
\end{array}$ 
\\
$w1,240,w3,x$ berechne x passend:\\ 
$\pi_1(w1) \circ \pi_2(240) \circ \pi_3(w3) \circ \pi_4(x) = id$\\
$(120 \circ w3) \circ 240 \circ \pi_4(x) = id$\\
$w1 \circ \pi_4(x) = id$\\
$\Rightarrow \pi_4(x) = w1$\\
$\Rightarrow x=w2$\\

\paragraph{Beispiel ISBN-Code:} Paritätscode mit Gewichten der Länge $n=10$ mit Basis $q=11$. Zeichen: $0,1,2,...,9,10=x$\\
$\begin{array}{ccccccccccc}
 & a1 & a2 & a3 & a4 & a5 & a6 & a7 & a8 & a9 & a10 \\ 
Gew & 10 & 9 & 8 & 7 & 6 & 5 & 4 & 3 & 2 & 1
\end{array}$
\\\\
Ergänze zu einem Code-Wort: 3-528-06783-a:\\
$3\cdot 10+5\cdot 9+2\cdot 8+8\cdot 7+0\cdot 6+6\cdot 5+7\cdot 4+8\cdot 3+3\cdot 2+a\cdot 1=0$\marginpar{mod 11}\\
$30+45+16+56+30+28+24+6+a=0$\\
$8+1+5+1+8+6+2+6+a=0$\\
$37+a=0$\\
$4+a=0 \Rightarrow a=7$

\section{Kryptographie}


% Nr. 2
\includegraphicsdeluxe{krypto1.jpg}{Verschlüsselung und Entschlüsselung (f, g sind öffentlich und k, $\tilde{k}$ geheim)}{fig:krypto1}

\paragraph{symmetrisch} $k=\tilde{k}$ oder $\tilde{k}=k$ kann aus k leicht berechnet werden. 
\paragraph{asymmetrisch} $k \neq \tilde{k}, \tilde{k}$ kann nicht oder nur sehr schwer berechnet werden.

\subsection{Symmetrische Verfahren}
\subsubsection{Stromchiffren}
Als Klartext nehmen wir eine Bitfolge. Der geheime Schlüssel auch. \marginpar{Nachteil des Verfahrens: langer Schlüssel}

% 3
\includegraphicsdeluxe{krypto2.jpg}{Funktionsweise symmetrische Verschlüsselung. Die Rote + Operation ist eigentlich eine - Operation. Bei Bits, also mod 2, kann aber Plus durch Minus ersetzt werden.}{fig:krypto2}



\paragraph{Beispiel:} Pseudo-Zufallsgenerator\\
\resizebox{\linewidth}{!}{% Graphic for TeX using PGF
% Title: /home/martin/Dokumente/hochschule/hm/ss15/dm/Mitschrift/pics/krypto_pseudo_zuf-generator.dia
% Creator: Dia v0.97.2
% CreationDate: Thu May 28 09:32:12 2015
% For: martin
% \usepackage{tikz}
% The following commands are not supported in PSTricks at present
% We define them conditionally, so when they are implemented,
% this pgf file will use them.
\ifx\du\undefined
  \newlength{\du}
\fi
\setlength{\du}{15\unitlength}
\begin{tikzpicture}
\pgftransformxscale{1.000000}
\pgftransformyscale{-1.000000}
\definecolor{dialinecolor}{rgb}{0.000000, 0.000000, 0.000000}
\pgfsetstrokecolor{dialinecolor}
\definecolor{dialinecolor}{rgb}{1.000000, 1.000000, 1.000000}
\pgfsetfillcolor{dialinecolor}
\definecolor{dialinecolor}{rgb}{1.000000, 1.000000, 1.000000}
\pgfsetfillcolor{dialinecolor}
\fill (1.715000\du,4.050000\du)--(1.715000\du,6.750000\du)--(12.785000\du,6.750000\du)--(12.785000\du,4.050000\du)--cycle;
\pgfsetlinewidth{0.100000\du}
\pgfsetdash{}{0pt}
\pgfsetdash{}{0pt}
\pgfsetmiterjoin
\definecolor{dialinecolor}{rgb}{0.000000, 0.000000, 0.000000}
\pgfsetstrokecolor{dialinecolor}
\draw (1.715000\du,4.050000\du)--(1.715000\du,6.750000\du)--(12.785000\du,6.750000\du)--(12.785000\du,4.050000\du)--cycle;
% setfont left to latex
\definecolor{dialinecolor}{rgb}{0.000000, 0.000000, 0.000000}
\pgfsetstrokecolor{dialinecolor}
\node at (7.250000\du,5.195000\du){Pseudo-};
% setfont left to latex
\definecolor{dialinecolor}{rgb}{0.000000, 0.000000, 0.000000}
\pgfsetstrokecolor{dialinecolor}
\node at (7.250000\du,5.995000\du){Zufallszahlengenerator};
\definecolor{dialinecolor}{rgb}{1.000000, 1.000000, 1.000000}
\pgfsetfillcolor{dialinecolor}
\fill (2.021250\du,8.900000\du)--(2.021250\du,11.600000\du)--(6.678750\du,11.600000\du)--(6.678750\du,8.900000\du)--cycle;
\pgfsetlinewidth{0.100000\du}
\pgfsetdash{}{0pt}
\pgfsetdash{}{0pt}
\pgfsetmiterjoin
\definecolor{dialinecolor}{rgb}{0.000000, 0.000000, 0.000000}
\pgfsetstrokecolor{dialinecolor}
\draw (2.021250\du,8.900000\du)--(2.021250\du,11.600000\du)--(6.678750\du,11.600000\du)--(6.678750\du,8.900000\du)--cycle;
% setfont left to latex
\definecolor{dialinecolor}{rgb}{0.000000, 0.000000, 0.000000}
\pgfsetstrokecolor{dialinecolor}
\node at (4.350000\du,10.045000\du){Schlüssel k};
% setfont left to latex
\definecolor{dialinecolor}{rgb}{0.000000, 0.000000, 0.000000}
\pgfsetstrokecolor{dialinecolor}
\node at (4.350000\du,10.845000\du){(Startwert)};
\pgfsetlinewidth{0.100000\du}
\pgfsetdash{}{0pt}
\pgfsetdash{}{0pt}
\pgfsetbuttcap
{
\definecolor{dialinecolor}{rgb}{0.000000, 0.000000, 0.000000}
\pgfsetfillcolor{dialinecolor}
% was here!!!
\pgfsetarrowsend{stealth}
\definecolor{dialinecolor}{rgb}{0.000000, 0.000000, 0.000000}
\pgfsetstrokecolor{dialinecolor}
\draw (4.350000\du,8.900000\du)--(4.482500\du,6.750000\du);
}
\pgfsetlinewidth{0.100000\du}
\pgfsetdash{}{0pt}
\pgfsetdash{}{0pt}
\pgfsetbuttcap
{
\definecolor{dialinecolor}{rgb}{0.000000, 0.000000, 0.000000}
\pgfsetfillcolor{dialinecolor}
% was here!!!
\pgfsetarrowsend{stealth}
\definecolor{dialinecolor}{rgb}{0.000000, 0.000000, 0.000000}
\pgfsetstrokecolor{dialinecolor}
\draw (12.785000\du,5.400000\du)--(20.500000\du,5.350000\du);
}
% setfont left to latex
\definecolor{dialinecolor}{rgb}{0.000000, 0.000000, 0.000000}
\pgfsetstrokecolor{dialinecolor}
\node[anchor=west] at (14.400000\du,4.850000\du){...,k3,k2,k1};
% setfont left to latex
\definecolor{dialinecolor}{rgb}{0.000000, 0.000000, 0.000000}
\pgfsetstrokecolor{dialinecolor}
\node[anchor=west] at (14.400000\du,5.650000\du){};
% setfont left to latex
\definecolor{dialinecolor}{rgb}{0.000000, 0.000000, 0.000000}
\pgfsetstrokecolor{dialinecolor}
\node[anchor=west] at (14.400000\du,6.450000\du){Schlüsselfolge};
\end{tikzpicture}
}

\paragraph{binäres Schieberegister} Wir rechnen mod 2
\begin{itemize}
\item Berechne $w=c_1 s_1+c_2 s_2+...+c_n s_n$
\item s1 ausgeben
\item Alle $s_i$ um eins nach rechts schieben
\item $s_n := w$
\end{itemize}
Der Schlüssel ist die erste Belegung von $s_1, ..., s_n$

\paragraph{Beispiel}


% 4
\includegraphicsdeluxe{krypto3.jpg}{Beispiel}{fig:krypto_beispiel}

Zellen gleiche Belegung: Dann geht es von vorne los $\rightarrow$ Der Schlüsselstrom ist periodisch. 
