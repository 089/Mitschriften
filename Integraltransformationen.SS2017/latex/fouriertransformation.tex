\part{Fourier-Transformation}
\section{Fourier-Transformation}
\section{Fourier-Transformation}
\subsection{Definition Fourier-Transformation}
\subsection{Definition Fourier-Transformation und inverse Fourier-Transformation und wichtige Beispiele, etwa Dirac-Impuls, Rechtecksimpuls, $ e^{iat} $}
\subsection{Rechenregeln und Beispielrechnungen}
\subsection{Fourier-Sinus-Transformation, Fourier-Cosinus-Transformation}
\subsection{Fourier-Integralsatz, Parcevalsche Gleichung}
\section{Fourier-Reihen}
\subsection{Darstellung periodischer Funktionen als Fourier-Reihen}
\subsection{Fourier-Koeffizienten (als „Inverse“)}
\subsection{Orthogonalitätsbeziehung}
\subsection{Darstellungssatz - analog Fourier-Integralsatz}
\subsection{Gibbsches Phänomen}
\subsection{Sinus- und Cosinusreihe für ungerade und gerade Funktionen}
\subsection{Rechenregeln und Analogie zur Fouriertransformation}
\subsection{Größenordnung der Fourier-Koeffizienten}
\subsection{Periodische Faltung}
\subsection{Zusammenhang zwischen Fourier-Transformation und Fourier-Reihen}
\section{Anwendungen in Signalverarbeitung und Kompression}
\subsection{Signalübertragung: Modulation und Multiplexing}
\subsubsection{Phasenmodulation mit QPSK – Quadrature Phase Shif Keying}
\subsection{FDM – Frequency Division Multiplexing}
\subsection{Digitale Filter und Ideen der mp3-Kompression}
\subsubsection{Faltung von Folgen, Faltung und Matrizenmultiplikation}
\subsubsection{Tiefpassfilter, Hochpassfilter}
\subsubsection{Kompression von Audiosignalen: Bandunterteilung und Idee der psychoakustischen Modelle}
\section{Diskrete Fourier-Transformation und Abtasten (Sampling)}
\subsection{Definition der diskreten Fourier-Transformation und Inverse}
\subsection{Abtasttheorem nicht-periodischer Fall}
\subsection{Abtasttheorem von Shannon-Nyquist für den periodischen Fall}
\subsection{Rezept zur Signalrekonstruktion}
\subsection{Fast Fourier Transform}