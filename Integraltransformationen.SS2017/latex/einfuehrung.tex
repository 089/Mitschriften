\part{Einführung}
\section{Funktionaltransformationen*}
Zur Einführung empfehle ich die Videos \textit{*** Integraltransformationen Einführung Teil 1} von Stephan Mueller \url{https://youtu.be/9JEuiY8mKOw} und \textit{20.08 Funktionaltransformationen, Fourier, Laplace, z} Jörn Loviscach \url{https://youtu.be/A6UK5cqSYic}.

Integraltransformationen sind spezielle Funktionaltransformationen.

\section{Zeitfunktionen}
\section{Dirac-Stoß}
\section{Faltung von Funktionen}
\section{komplexe Funktionen}
\subsection{insbesondere Exponentialfunktion, Logarithmus, sin, cos, sinh, cosh, Wurzel, Polynome}
\subsection{Grenzwerte für komplexe Funktionen}
\subsection{Stetigkeit}
\subsection{Differenzierbarkeit, Ableitung}
\subsection{Holomorphie}
\subsection{Cauchy-Riemannsche Differentialgleichungen}
\subsection{Potenzreihen, Konvergenz, Konvergenzradius}
\subsection{spezielle Laurentreihe und Laurentreihe}
\subsection{Potenzreihen und holomorphe Funktionen}
\subsection{Taylorentwicklung}
\subsection{Laurentreihen und Holomorphie}
\subsection{Stammfunktionen und Kurvenintegrale}
\subsection{Wegunabhängigkeit von Kurvenintegralen}
\subsection{Hauptsatz der Integralrechnung}
\subsection{Cauchyscher Integralsatz}
\subsection{Laurent-Koeffizienten und Kurvenintegrale}
\subsection{Residuum und Polstellen}
\subsection{Residuensatz und Anwendung zur Berechnung von Kurvenintegralen}