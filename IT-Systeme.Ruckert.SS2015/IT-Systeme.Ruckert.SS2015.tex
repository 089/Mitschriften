% Basis
\documentclass[a4paper]{article}
\usepackage[ngerman]{babel}
\usepackage[utf8]{inputenc}

% Libs
\usepackage{graphicx} % Bilder
%\usepackage{mmixtex/mmix}	% MMIX
\usepackage{listings}	% Programmcode
\usepackage{tikz}	% Für Diagramme aus Dia

% Verweise anklickbar machen
\usepackage{hyperref} % 1. 
\usepackage[figure]{hypcap} % 2. 

% EIGENE BEFEHLE -----------------------------------------------
% Datum der einzelnen Lektionen definieren
\newcommand{\ldate}{2015-05-01}	% define lessiondate

% Bildbreite: Originalgröße oder falls größer als Seitenbreite/Spalte ==> skalieren
\makeatletter
\def\ScaleIfNeeded{%
\ifdim\Gin@nat@width>\linewidth
\linewidth
\else
\Gin@nat@width
\fi
}
\makeatother

% Neuer Befehl der Bilder standartisiert einbindet:
% includegraphicsdeluxe benötigt \ScaleIfNeeded und \ldate
% 1. innerhalb der figure-Umgebung mit dem Versuch, das Bild mittels [!htb] an genau dieser Position einzufügen
% 2. mit der Original- oder einer skalierten (mittels ScaleIfNeeded s.o.) Größe
% 3. mit einer Beschriftung (\caption{})
% 4. mit einem Label (\label{})
\newcommand{\includegraphicsdeluxe}[3]{
	\begin{figure}[!htb] 
	\centering
	\includegraphics[width=1\ScaleIfNeeded]{pics/\ldate_#1}
	\caption{#2}
	\label{#3}
	\end{figure}
}

% AND und OR Symbole
\newcommand{\und}{\wedge}
\newcommand{\oder}{\vee}

% Grundmengen
\newcommand{\N}{\mathbb{N}}
\newcommand{\Z}{\mathbb{Z}}
\newcommand{\Q}{\mathbb{Q}}
\newcommand{\R}{\mathbb{R}}
\newcommand{\C}{\mathbb{C}}



%Kopf- und Fußzeile
\usepackage{fancyhdr}
\pagestyle{fancy}
\fancyhf{}
 
%Kopfzeile mittig mit Kaptilname
\fancyhead[L]{\textsf{\nouppercase{\leftmark}}}
\fancyhead[R]{\textsf{\thepage}}
%Linie oben
\renewcommand{\headrulewidth}{0.5pt}
%Fußzeile links bzw. innen
\fancyfoot[C]{\tiny{M. Zell}}
\fancyfoot[L]{\tiny{IT-Systeme}}
\fancyfoot[R]{\tiny{SS 2015}}
%Linie unten
\renewcommand{\footrulewidth}{0.5pt}
 
% Fußzeile auf jeder Seite - auch Kapitel und Inhaltsverzeichnis
\fancypagestyle{plain}{%
   \fancyhf{}%
	\fancyfoot[C]{\tiny{M. Zell}}
	\fancyfoot[L]{\tiny{IT-Systeme}}
	\fancyfoot[R]{\tiny{SS 2015}}
   \renewcommand{\headrulewidth}{0.0pt} %obere Linie ausblenden
}

\author{M. Zell}
\title{Mitschrift\\IT-Systeme SS2015\\Prof. Dr. Martin Ruckert}

\begin{document}
\maketitle
\tableofcontents

% Vorlesung vom 06.05.2015
\renewcommand{\ldate}{2015-05-06}	% define lessiondate
\fancyfoot[R]{\tiny{SS 2015 / \ldate}}

\section{Zahlendarstellungen und Konventionen} 

Heute ging es um die Zahlendarstellung und Konventionen in MMIX. Ein Schwerpunkt war Umwandlung von Zahlen in unterschiedliche Zahlensysteme, insbesondere:
\begin{itemize}
\item Binär $ \rightarrow $ Dezimal $ \rightarrow $ Binär
\item Binär $ \rightarrow $ Hexadezimal $ \rightarrow $ Binär
\end{itemize}
Diese Umwandlung ist hauptsächlich für Input und Output wichtig. 

\subsection{Binär $ \rightarrow $ Hexadezimal}

Eine Speicherzelle in MMIX (Octa) hat 64 Bit. Je 4 Bit (Nibble, halbes Byte) entsprechen einer hexadezimalen Ziffer. \marginpar{Beispiel: $ 1100_{2} = C_{16} $ $ 1101_{2} = D_{16} $} Die Umwandlung einer Speicherzelle - ein Byte pro [] - kann also direkt erfolgen: [.][.][.][10110011] $ \rightarrow $ [.][.][.][B3]
In MMIX kann man die Ziffern mittels SLU und AND \#F extrahieren. Man kann auf verschiedene Arten ein Zeichen in ASCII-Code umwandeln:

\subsubsection{Mit einem Branch}

\marginpar{
CMP mit 10\\
PBN $ \rightarrow $ + '0'\\
sonst + ('A'-10)
}

\begin{tabular}{|c|c|}
\hline Wert x & ASCII-Code \\ 
\hline 0 & '0' + x \\ 
\hline 1 & '0' + x \\ 
\hline \vdots & \vdots \\ 
\hline 9 & '0' + x \\ 
\hline 10 & 'A' - 10 + x \\ 
\hline 11 & 'B' - 10 + x \\ 
\hline \vdots & \vdots \\ 
\hline 15 & 'F' - 10 + x \\ 
\hline 
\end{tabular}

\subsubsection{Mit einer Tabelle} 
\begin{lstlisting}
Tabelle BYTE "0123456789ABCDEF"
	LDA tmp,Tabelle
	LDBU $X,tmp,ziffer
\end{lstlisting}

\subsubsection{Laufzeitvergleich}
Mit der Tabelle werden ca. 4 Zyklen pro Ziffer benötigt, mit einem Branch $ 4 + \frac{10 * 1 + 5 * 3}{15} $ Zyklen.

\subsection{Hexadezimal $ \rightarrow $ Binär}
Gegeben sind ASCII-Codes 0,...,9,A,...,F,a,...,f. Diese können ebenfalls mittels Tabelle umgewandelt werden oder mit CMP, Branch und anschließendem Zusammensetzen mittels SRU, OR (Laufzeit: 4 Zyklen pro Ziffer)

\subsection{Dezimal $ \rightarrow $ Binär}
Jetzt soll "12345" von hinten nach vorne umgewandelt werden. Als erstes werden die Zeichen in eine Zahl umgewandelt: $ Zeichen * 10^{Stelle}, Stelle \in Z $. Mit dem Hornerschema: sieht das für das Beispiel wie folgt aus: $ (((1 * 10 + 2)* 10 + 3) * 10 + 4) * 10 + 5 $

\subsubsection{Pseudocode}
\begin{verbatim}
Schleife:
	ASCII Code lesen
	'0' subtrahieren
	Mit 10 multiplizieren
	ASCII code lesen
	'0' subtrahieren
	Addieren ...
\end{verbatim}
	
\subsubsection{MMIX Code}
Jetzt geht es darum in MMIX die Umwandlung von Dezimalzahlen in binäre Zahlen zu erledigen. Als erstes wandeln wir nur ein einziges Zeichen um:
\begin{lstlisting}
	PREFIX :ToBinary:
str IS $10	% String mit Dezimalziffern
d IS $1
% Returnwert die entsprechende Zahl

% Version 1: nur ein Zeichen z.B. "5"
:ToBinary	LDBU d,str,0
	SUB d,d,'0'
	SET $0,d
	POP 1,0
\end{lstlisting}

Nun wandeln wir genau zwei Zeichen um:	
\begin{lstlisting}
% Version 2: genau zwei Zeichen z.B. "04" oder "15"
n IS $2
:ToBinary	LDBU d,str,0
	SUB d,d,'0'			% 1. Stelle
	MUL n,d,10
	LDBU d,str,1
	SUB d,str,1
	SUB d,d'0'
	ADD n,n,d
	
	SET $0,n
	POP 1,0
\end{lstlisting}	

Nun eines oder zwei. Der String endet mit einem Nullbyte!
\begin{lstlisting}
% Version 3: Ein oder zwei Zeichen, String endet mit einem Null Byte
z IS $3
:ToBinary	LDBU d,str,0
	SUB z,d,'0'		% 1. Stelle
	
	LDBU d,str,1
	BZ d,single		% single digit
	
	MUL n,z,10
	SUB z,d,'0'
	ADD n,n,z
	
	SET $0,n
	POP 1,0			

single:		SET $0,z
	POP 1,0
\end{lstlisting}

1. Optimierung. Eine Variable weniger. Das z brauchen wir nicht mehr. 
\begin{lstlisting}	
% Version 3b: Ein oder zwei Zeichen, String endet mit einem Null Byte
str IS $0
d IS $1
n IS $2
:ToBinary	LDBU d,str,0
	SUB n,d,'0'		% 1. Stelle
	
	LDBU d,str,1
	BZ d,finish		% single digit
	
	MUL n,n,10
	SUB d,d,'0'
	ADD n,n,d
	
finish:		SET $0,z
	POP 1,0
\end{lstlisting}
	
Nun wandeln wir beliebig viele Zeichen eines Strings um. 
\begin{lstlisting}	
% Version 4: mit beliebig vielen Ziffern; endet immer mit Nullbyte
str IS $0
d IS $1
n IS $2

:ToBinary	SET n,0

Loop:		LDBU d,str,0
	BZ d,finish
	ADD str,str,1
	MUL n,n,10
	SUB d,d,'0'
	ADD n,n,d
	JMP loop
			
finish:		SET $0,n
	POP 1,0
\end{lstlisting}
	
Indem wir das Programm umstellen können wir auf den JMP verzichten, was einen Zyklus einspart. 
\begin{lstlisting}	
% Version 5: Optimierung der Schleife
% Laufzeit 15 k + 6 + 2
str IS $0
d IS $1
n IS $2

:ToBinary	SET n,0
	JMP 1F

Loop:		ADD str,str,1
	MUL n,n,10
	SUB d,d,'0'
	ADD n,n,d

1H			LDBU d,str,0
	PBNZ d,Loop
			
finish:		SET $0,n
	POP 1,0
\end{lstlisting}

Nun ersetzen wir den zyklen-intensiven Befehl MUL (ca. 10 Zyklen)
\begin{lstlisting}	
% Version 6: Ersatz fuer MUL
% Laufzeit 7 k + 6 + 2
str IS $0
d IS $1
n IS $2

:ToBinary	SET n,0
	JMP 1F

Loop:		ADD str,str,1

	% Multiplikation mit 10
	4ADDU n,n,n
	SL n,n,1

	SUB d,d,'0'
	ADD n,n,d

1H			LDBU d,str,0
	PBNZ d,Loop
			
finish:		SET $0,n
	POP 1,0
\end{lstlisting}

\subsubsection{Exkurs xADDU-Befehle} 
Die 2ADDU/4ADDU/8ADDU/16ADDU-Befehle benutzen wir hier für eine schnelle Alternative zur Multiplikation mit MUL. Sonst kann man diese Befehle z.B. für die Adressierung: Index * 2/4/8/16 + Basis nutzen. 
Beispiel: 
\begin{lstlisting}
4ADDU n,n,n	n <-- 4n+n (Multiplikation mit 4)

2ADDU $X,$Y,$Z	$X <-- 2 * $Y + $Z
4ADDU $X,$Y,$Z	$X <-- 4 * $Y + $Z
8ADDU $X,$Y,$Z	$X <-- 8 * $Y + $Z
16ADDU $X,$Y,$Z	$X <-- 16 * $Y + $Z
\end{lstlisting}

\subsection{Binär $ \rightarrow $ Dezimal mit Division}
Nun geht es um die Umwandlung von Zahlen in Strings (z.B.12345 nach "12345"). Hier benötigen wir die Division. Diese ist in MMIX mit ca. 60 Zyklen sehr teuer. Dennoch führt sie zum Ziel. 

\begin{lstlisting}
DIV n,n,10 % teuer ca. 60 Zyklen
GET d,rR % rR Register ist mit dem Rest der letzten Division gefuellt
ADD d,d,'0'
\end{lstlisting}

\subsection{Binär $ \rightarrow $ Dezimal mit Multiplikation}
Weil die Division teuer ist, kann man stattdessen multiplizieren. Die Idee ist, dass der Zahlenbereich eingeschränkt wird. \marginpar{[123456788].[00...0]} In den oberen 32 Bit ist die Zahl die umgerechnet werden soll und in den unteren lauter Nullen. Die Zahl n ($ n < 10^9 $) wird ein einziges Mal geteilt ($ n / 10^9 $). Dadurch wird erreicht, dass die Zahl im unteren 32-Bit-Block landet. Nun wird mit 10 Multipliziert. \marginpar{[0].[123456789] * 10 = [1].[23456789]} Dadurch gelangt die vorderste Ziffer im oberen 32-Bit-Block. Diese wird ausgegeben und anschließend mittels ADDMH \#F gelöscht. \marginpar{[0].[23456789] * 10 = [2].[3456789]} Dieser Vorgang wird nun wiederholt. Die Laufzeit beträgt nun ca. 60 + 6K für k Ziffern statt 60 k.
% Vorlesung vom 20.05.2015
\renewcommand{\ldate}{2015-05-20}	% define lessiondate
\fancyfoot[R]{\tiny{SS 2015 / \ldate}}

\section{Synchronisierung von Threads}

\subsection{Vorteile und Eigenschaften}
\begin{itemize}
\item gemeinsamer Speicher (Adressraum)
\item getrennte Register
\item getrennter Stack
\end{itemize}

\subsection{Wiederholung}
Der Wechsel zwischen Threads passiert mit SAVE (Alles kommt in den Stack; auf der obersten Position bleibt die Adresse der Daten) und UNSAVE (Nimmt die Adresse und holt die Daten vom Thread). Threadwechsel sind im Vergleich zu Prozesswechseln billig. 

\paragraph{Problem:} buffered IO. Der eine Thread speichert, liest und kopiert der andere verarbeitet in den Lesephasen:\\ 
\resizebox{\linewidth}{!}{\input{pics/\ldate_threads_buffered_io}}

\subsection{Synchronisierung}

\subsubsection{Einfachster Fall:} Zwei Threads mit einem gemeinsamen Buffer:
\input{pics/\ldate_threads_gemeinsamer_buffer.tex}

\paragraph{Synchronisation mit einer Semaphore} ("Licht", "Ampel")
\begin{lstlisting}
***
\end{lstlisting}

\paragraph{Sequentiell inkonsistent:} 

\begin{figure}[htbp]
\includegraphics[width=1\textwidth]{pics/\ldate_sequ_inkonsistenz.jpg}
\caption{Beispiel zur sequentiellen Inkonsistenz}
\label{fig:sequ_inkonsistenz}
\end{figure}

Zur Abhilfe gibt es den Befehl SYNC. SYNC 3 bis SYNC 7 sind \textbf{privileged} (Abb. \ref{fig:sequ_inkonsistenz}). 

\begin{tabular}{|c|c|}
\hline SYNC & Bezeichnung \\ 
\hline 0 & drain pipeline \\ 
\hline 1 & finish all stores before starting following stores \\ 
\hline 2 & finish all loads before starting following loads \\ 
\hline 3 & beides; 1 und 2 \\ 
\hline 4 & Power Save Mode (z.B. bei Wartezyklen) \\ 
\hline 5 & empty WriteBuffers and caches to memory \\ 
\hline 6 & clear TLB (Translation Lookasside Buffers, d.h. Cache für page tables) \\ 
\hline 7 & Clear all caches \\ 
\hline 
\end{tabular} 

\subsubsection{Double Buffering (Abb. \ref{fig:double_buffering}) und Buffer Swapping (Abb. \ref{fig:buffer_swapping})}

\begin{figure}[htbp]
\includegraphics[width=1\textwidth]{pics/\ldate_double_buffering.jpg}
\caption{Double Buffering}
\label{fig:double_buffering}
\end{figure}

\begin{figure}[htbp]
\includegraphics[width=1\textwidth]{pics/\ldate_buffer_swapping.jpg}
\caption{Buffer Swapping}
\label{fig:buffer_swapping}
\end{figure}

\begin{lstlisting}

	% Semaphore,Adresse von Buffer, Adresse von anderer Semaphore
S1	OCTA 	1,Buffer1,S2
S2	OCTA 	0,Buffer2,S1

Consumer:
tmp IS $0
buffer IS $1
S GREG 0

	LDA s,S1	Initialisieren

1H	LDO tmp,s,0
		BZ tmp,1B
		SYNC 2
		LDO buffer,s,8
		%.... Use Buffer/lesen
	
		STCO 0,s,0	Release
		LDO s,s,16	Buffer swap
\end{lstlisting}

Diese Synchronisation erfordert, dass der Thread, der den Speicher freigibt, bestimmt, wem der Speicher als nächstes gehört. Falls unklar ist welcher Thread als nächstes Zugriff braucht macht man folgendes. Dabei steht die 0 für \glqq Ressource ist frei verfügbar\grqq und 1 für \glqq Ressource wird gerade verwendet\grqq:

\begin{lstlisting}
S	OCTA	0
\end{lstlisting}

\paragraph{Naive Lösung} Eine naive Lösung könnte wie folgt aussehen, allerdings gibt es ein \textbf{Problem:} Beide wollen auf die Liste zugreifen. Es kann also passieren, dass beide gleichzeitig auf S zugreifen. Man braucht daher eine Operation die in einem Schritt (ununterbrechbar) ein OCTA liest, vergleicht und schreibt. Das ist bei MMIX die CSWAP \$X,\$Y,\$Z Instruktion. 

\begin{lstlisting}
***
1H	LDO	$0,S	Warte bis S=0
	BNZ $0,1B
	SYNC 2
	STCO 1,S
	... 
	Use Release
	SYNC 1
	STCO 0,	Release
\end{lstlisting}

% Vorlesung vom 27.05.2015
\renewcommand{\ldate}{2015-05-27}	% define lessiondate
\fancyfoot[R]{\tiny{SS 2015 / \ldate}}


\subsection{Fortsetzung Synchronisation von Threads und Prozessen}

\paragraph{CSWAP Instruktion (Compare and Swap):}

\begin{lstlisting}
CSWAP $X,$Y,$Z	% adresse = $Y+$Z
\end{lstlisting}

Vergleiche M[adresse] mit rP. \marginpar{rP: Prediction register}
Falls gleich:  
\begin{lstlisting}
speichern: M[adresse] $ <-- $ $X
setze: $X <-- 1
\end{lstlisting}
sonst:
\begin{lstlisting}
lade: rP <-- M[adresse]
setze: $x <-- 0
\end{lstlisting}
Die Instruktion ist atomar, d.h. \textbf{ununterbrechbar}.

\textbf{Gemeinsame Daten}

\begin{figure}[htbp]
% Nr. 1
\includegraphics[width=1\textwidth]{pics/\ldate_gemeinsame_daten.jpg}
\caption{Gemeinsame Daten}
\label{fig:gemeinsame_daten}
\end{figure}

\begin{lstlisting}
Aquire:	falls S == 0, setze S auf 1 (blocking)
	bearbeite buffer (protected code)

Release: setze S auf 0
\end{lstlisting}
Was nicht geht:
\begin{lstlisting}
1H	LDO $0,S
	BNZ $0,1B	
	STCO 1,S
\end{lstlisting}

\textbf{mit CSWAP}
\begin{lstlisting}
1H	PUT rP,0
	SET $0,1
	CSWAP $0,S	% atomare Operation
	BZ $0,1B
\end{lstlisting}


\subsection{Strategien zum Schreiben in den Cache}
\begin{figure}[htbp]
% Nr. 2
\includegraphics[width=1\textwidth]{pics/\ldate_speicherhirarchie.jpg}
\caption{Speicherhirarchie in der CPU}
\label{fig:speicherhirarchie}
\end{figure}
\marginpar{ganz viele Erklärungen von Herrn Ruckert, die v.a. parallel zum Abmalen des Tafelbilds nicht aufgeschrieben werdne können.}

\paragraph{Write allocate} Die passende cacheline wird geladen, der Wert kommt in die cacheline.
\paragraph{Write Back} Die passende cacheline wird geladen, geändert und geschrieben.
\paragraph{Write through} Falls die passende cacheline vorhanden ist, wie write back, sonst wird der Wert am cache vorbei in die nächsttiefere Speicherebene geschrieben (vgl. Abb. \ref{fig:speicherhirarchie}).

\subsection{Funktionsweise von Caches}
Caches dienen dem schnellen Speicherzugriff. Meist sind sie direkt am Chip (L1-Caches). L2-Caches sind im gleichen Gehäuse der CPU. Es gibt sogenannte cachelines (z.B. 64 Byte). Es wird immer die komplette cacheline geladen oder gespeichert, nie einzelne Bytes (Abb. \ref{fig:cachlines}).  

\begin{figure}[htbp]
% Nr. 3
\includegraphics[width=1\textwidth]{pics/\ldate_cachelines.jpg}
\caption{Cachelines}
\label{fig:cachlines}
\end{figure}

\subsection{Non Blocking Synchronisation}
\paragraph{wait-free:} Jede Operation wird in endlich vielen Schritten fertig. 
\paragraph{lock-free:} Irgendeine Operation wird in endlich vielen Schritten fertig. 
\paragraph{obstruction-free:} Ein Thread wird in endlich vielen Schritten fertig, wenn alle anderen Threads gerade pausieren. 

\subsection{Cacheprotokolle MESI}
\begin{figure}[htbp]
% Nr. 4
\includegraphics[width=1\textwidth]{pics/\ldate_cache_mesi.jpg}
\caption{Funktionsweise von Cache MESI}
\label{fig:cache_mesi}
\end{figure}

\section{Datenstrukturen}
Bisher haben wir z.B. Arrays kennengelernt. Jetzt schauen wir uns dynamische Datenstrukturen an. Das sind Datenstrukturen, die zur Laufzeit die Größe ändern. 
\paragraph{Dynamische Speicherverwaltung} z.B. die Funktion malloc(size) in C.\marginpar{memmory allocate} Diese Funktion gibt die Adresse zurück, an der size BYTE verfügbar sind. Die Funktion free(adresse) gibt den Speicher an der angegebenen Adresse wieder frei. 

\subsection{Verkettete Listen} 

\begin{figure}[htbp]
% Nr. 5
\includegraphics[width=1\textwidth]{pics/\ldate_verkettete_listen.jpg}
\caption{verkettete Listen}
\label{fig:verkettete_listen}
\end{figure}

\paragraph{Insert:} Verkettete Liste mit NEXT (Adresse des nächsten Knotens) und VALUE (Wert 1 OCTA). 

\begin{figure}[htbp]
% Nr. 6
\includegraphics[width=1\textwidth]{pics/\ldate_llist_insert.jpg}
\caption{insert bei verketteten Listen}
\label{fig:llist_insert}
\end{figure}

\begin{lstlisting}
	PREFIX :Insert:
Head IS $0	% Adresse des list heads (param 1)
NewElement IS $1	% Adresse des neuen Elements (param 2)
tmp IS $2

:Insert
	LDOU tmp,Head,0
	STOU NewElement,Head,0
	STOU tmp,NewElement,0
	POP 0,0
\end{lstlisting}

Variante 2
\begin{lstlisting}
	PREFIX :Insert:
Head IS $0	% Adresse des list heads (param 1)
NewElement IS $1	% Adresse des neuen Elements (param 2)
tmp IS $2

:Insert
	LDOU tmp,Head,0
	STOU tmp,NewElement,0
	STOU NewElement,Head,0
	
	POP 0,0
\end{lstlisting}

\paragraph{Delete:} 

\begin{lstlisting}
	PREFIX :Delete:
Head IS $0	% Adresse des list heads (@param 1)
% Rueckgabewert: 
%	Null: wenn die Liste leer ist
%	sonst: Adresse des geloeschten Elements
return IS $1
tmp IS $2

:Delete LDOU return,Head,0
	BZ return,1F
	LDOU tmp,return,0
	STOU tmp,Head,0
1H	SET $0,return
	POP 1,0
\end{lstlisting}
% Vorlesung vom 03.06.2015
\renewcommand{\ldate}{2015-06-03}	% define lessiondate
\fancyfoot[R]{\tiny{SS 2015 / \ldate}}

\section{Pipelineprozessoren (Superskalar)}

\subsection{Aufbau Pipelineprozessor}
Bedingte Sprünge (Branch) stellen bei Pipelineprozessoren ein Problem dar. Dafür gibt es die sogenannte \textbf{Branch Prediction}, also das Raten ob der Branch genommen wird. In MMIX gibt es dazu Probable Branches (PBZ statt BZ) mit denen der Assembler auf einen sehr wahrscheinlichen Branch hingewiesen werden könne.  
\begin{figure}[htbp]
% Nr. 1 
% \includegraphics[width=1\textwidth]{pics/\ldate_aufbau_prozessor.jpg}
\caption{Aufbau Prozessor}
\label{fig:aufbau_prozessor}
\end{figure}

\subsection{Beispiel: Quicksort} \marginpar{Am besten in Wikipedia nachlesen.}
Man hat ein Array $\{7,3,9,2,8,6,5,7,9,1,4\}$ von Daten, die man sortieren will. Die Idee ist, dass man partitioniert, also links die kleinen und rechts die großen Werte sortiert. Rekursiv werden dann die kleinen Werte und die großen Werte sortiert. Um zu entscheiden was groß und was klein ist verwendet man das sogenannte \textbf{Median of Three (Mittlere von drei Werten)}. Im Beispiel ist dies die Zahl 6. 
\paragraph{Sortieren der drei Elemente} $\{7,3,9,2,8,6,5,7,9,1,4\} \rightarrow \{4,6,9,2,8,3,5,7,9,1,7\}$ Diesen Schmarrn macht man solange bis das Array so aussieht: $\{1,2,3,4,5,6,7,7,8,9,9\}$

\paragraph{Der innerste Loop} 

\begin{lstlisting}
1H	add i <-- i+1
start: load Ki	Lade Kex i
	vergleiche Ki < K
	if Ki < K goto 1H (oben)
	
2H	sub j <-- j-i
	load Kj
	verleiche K < Kj
	if K < Kj goto goto 2H (oben)
	
	if i < j tausche Ki und Kj und goto 1H

	sonst fertig
\end{lstlisting}


% Vorlesung vom 10.06.2015
\renewcommand{\ldate}{2015-06-10}	% define lessiondate
\fancyfoot[R]{\tiny{SS 2015 / \ldate}}

\section{Verwaltung des dynamischen Speichers zur Laufzeit}
Bei Objekten gibt es Konstruktoren und Destruktoren. In Sprachen wie C gibt es malloc, free. Im Assembler gibt es meistens ein Pool Segment (Intel: eXtra Segment). Hier wird ein sogenannter Heap eingerichtet, der Speicher reserviert und freigibt. \\
Wird ein Programm mit Argumenten gestartet stehen in \$0 die Anzahl der Argumente und in \$1 4...8 die Adresse des ersten Arguments. Die Werte der Argumente selbst stehen im Pool Segment 0x4000...0.

\subsection{Im Poolsegment dynmaisch Speicher anfordern}
Die Funktion \textbf{malloc}\index{malloc} soll erstellt werden (Parameter: Anzahl BYTE, Rückgabewert: Adresse wo diese BYTE verfügbar sind.):
\begin{itemize}
\item Rückgabe wert = erstes OCTA im Poolsegment.
\item Anzahl BYTES auf erstes OCTA aufaddieren
\item Anzahl BYTES vorher auf ein Vielfaches von acht aufrunden (wegen Alignment\index{Alignment}).
\end{itemize}

\begin{lstlisting}
	LOC #100
tmp	IS	$0
Main	SET	tmp+1,22
	PUSHJ	tmp,Malloc
	STCO	22,tmp
	SET	tmp+1,8
	PUSHJ	tmp,Malloc
	STCO	8,tmp,0
	TRAP	0,Halt,0
	
	PREFIX :Malloc:		% Version 1
size	IS	$0
base	IS	$1
free	IS	$2
tmp		IS	$3
:Malloc SETH	base,#4000
	LDOU	free,base,0
	ADD	size,size,7
	ANDN	size,size,7
	ADDU	tmp,free,size
	STOU	tmp,base,0
	SET	$0,free 
	POP	1,0	
\end{lstlisting}

\subsection{Beispielanwendung}
Wir speichern Nodes die aus zwei OCTA bestehen, einem Zeiger NEXT und einem Wert VALUE. Wir wollen zwei Funktionen New und Old haben. \textbf{Old} hat einen Parameter (Knoten) und fügt den Knoten der Liste mit alten Knoten hinzu. \textbf{New} prüft, ob alte Knoten vorhanden sind und wenn ja, ob diese genutzt werden können. Sonst nutzt man Malloc und reserviert sich so neuen Speicher für ein neuen Knoten (Node). 

\begin{lstlisting}
	LOC Data_Segment
	GREG	@
OldNodes	OCTA	0	% derzeit leer

	LOC #100
tmp	IS	$0
NEXT	IS	0
VALUE 	IS	8
Main	PUSHJ	tmp,New
	STCO	22,tmp,VALUE
	
	PUSHJ	tmp,New	
	STCO	8,tmp,VALUE
	
	SET	tmp+1,tmp
	PUSHJ	tmp,Old
	
	PUSHJ	tmp,New
	STCO	200,tmp,VALUE
	
	TRAP	0,Halt,0	% Ende Gelaende
	
	PREFIX :Malloc:		% Version 1
size	IS	$0
base	IS	$1
free	IS	$2
tmp		IS	$3
:Malloc SETH	base,#4000
	LDOU	free,base,0
	ADD	size,size,7
	ANDN	size,size,7
	ADDU	tmp,free,size
	STOU	tmp,base,0
	SET	$0,free 
	POP	1,0	
	
	PREFIX :Old:
Node	IS	$0	% Parameter
First	IS	$1

:Old	LDOU	First,:OldNodes % laedt Zeiger auf alten Knoten
	STOU	First,Node,:NEXT
	STOU	Node,:OldNodes
	POP	0,0
	
	PREFIX :New:
rJ	IS	$0
First	IS	$1	
tmp	IS	$2
:New	LDOU	First,:OldNodes
	BZ	First,1F	% nichts drin
	
	LDOU	tmp,First,:NEXT	% laden des Nextpointers vom 1. Knoten
	STOU	tmp,:OldNodes	% speichern
	SET	$0,First
	POP	1,0
	
1H	GET	rJ,:rJ
	SET	tmp+1,16	% Parameter Groesse = 10 uebergeben
	PUSHJ	tmp,:Malloc
	PUT	:rJ,rJ
	SET	$0,tmp
	POP	1,0

\end{lstlisting}

\subsection{Verschiedene Funktionen auf verketteten Listen}

\subsubsection{Einfügen am Anfang}
Haben wir bereits betrachtet.

\subsubsection{Löschen am Anfang}
Haben wir bereits betrachtet.

\subsubsection{Einfügen am Ende}

\paragraph{Version 0}
\begin{lstlisting}
:insertEnde LDOU	last,Head,0
0H	LDOU	tmp,last,:NEXT
	BZ	tmp,1F
	
	SET	last,tmp
	JMP	0B
	
1H	STOU	Node,Last,:NEXT
	POP	0,0	
\end{lstlisting}

\paragraph{Version 1}
\begin{lstlisting}
:insertEnde SET	last,Head
0H	LDOU	tmp,last,:NEXT
	BZ	tmp,1F
	
	SET	last,tmp
	JMP	0B
	
1H	STOU	Node,Last,:NEXT
	POP	0,0	
\end{lstlisting}

\paragraph{Version 2}
\begin{lstlisting}
:insertEnde	LDOU	tmp,Head,:NEXT
	BZ	tmp,1F
	
	SET	Head,tmp
	JMP	:InsertEnde
	
1H	STOU	Node,Head,:NEXT
	POP	0,0	
\end{lstlisting}

\paragraph{Version 3}
\begin{lstlisting}
Nodes	IS	$0
Head	IS	$1
tmp	IS	$2

:InsertEnd	LDOU	tmp,Head,:NEXT
	BZ	tmp,1F
	SET	Head,tmp
	JMP	:InsertEnd
1H	STOU	Node,Head,:NEXT
	POP	0,0	
\end{lstlisting}

\paragraph{Version 4}
\begin{lstlisting}
Nodes	IS	$0
Head	IS	$1
tmp	IS	$2
1H	SET	Head,tmp
:InsertEnd	LDOU	tmp,Head,:NEXT
	PBNZ	tmp,1B
	STOU	Node,Head,:NEXT
	POP	0,0	
\end{lstlisting}

\paragraph{Version 5 mit Loop doubling}
\begin{lstlisting}
Nodes	IS	$0
Head	IS	$1
tmp	IS	$2
1H	SET	Head,tmp
:InsertEnd	LDOU	tmp,Head,:NEXT
	PBNZ	tmp,1B
	STOU	Node,Head,:NEXT
	POP	0,0	
\end{lstlisting}

% Vorlesung vom 17.06.2015
\renewcommand{\ldate}{2015-06-17}	% define lessiondate
\fancyfoot[R]{\tiny{SS 2015 / \ldate}}

\section{Fragen von Studenten. Antworten vom Professor}

\subsection{Wie funktioniert loop unrolling?}

\begin{lstlisting}
	PREFIX :InsertEnd:
new	IS $0	Adresse des neuen Knoten
head	IS $1	Adresse von der Adresse des ersten Knotens
next	IS $2
NEXT	IS 0	Offset des NEXT-Feldes

% Variante 1
1H	SET head,next
:InsertEnd	LDOU	next,head,NEXT
	BNZ	next,1B
	STOU	new,head,NEXT
	POP	0,0
	
% Variante 2	
1H	SET head,next
:InsertEnd	LDOU	next,head,NEXT
	BZ	next,0F
	SET	head,next
	LDOU	next,head,NEXT
	BNZ	next,1B
	STOU	new,head,NEXT
	POP	0,0
	
OH	STOU	new,head,NEXT	
\end{lstlisting}

\includegraphicsdeluxe{foto_optimierung_code.jpg}{Optimierung mittels loop doubleing; Rot: neuer Wert, Gelb: Verwendung}{fig:foto_optimierung_code} % Nr. FOTO!

\paragraph{copy propagation} Statt der Kopie verwendet man das Original. Der optimierte Code lautet dann:

\begin{lstlisting}
:InsertEnd LDOU	next,head,NEXT
	BZ	next,0F

	LDOU	head,next,NEXT
	PBNZ	head,:InsertEnd
	STOU	new,next,NEXT
	POP	0,0
	
0H	STOU	new,head,NEXT
	POP	0,0	
\end{lstlisting} 

\subsection{Wie kann man die Anzahl von Bytes auf ein Vielfaches von 8 aufrunden?}

\subsubsection{Wie geht das Dezimal?}
\begin{itemize}
\item Abrunden: $1234567 \rightarrow 1234000$
\item Aufrunden: $1234567 \rightarrow 12345000$
\item Aufrunden durch Abrunden: Erst $+999$, dann abrunden: $1234557 + 999 = 12345566 \rightarrow 12345000$
\end{itemize}

\subsubsection{Wie geht das binär?}
Abrunden auf Vielfaches von 8 (Abb. \ref{fig:abrunden_auf_vielfaches_von_8})
\begin{lstlisting}
AND	size,size,7
(AND	size,size,#FFF....F8 % wg. Groesse nicht moeglich)
ANDN	size,size,7 % Ersatz fuer Zeile oben drueber
\end{lstlisting}
\includegraphicsdeluxe{abrunden_auf_vielfaches_von_8.jpg}{Abrunden auf Vielfaches von 8 (binär)}{fig:abrunden_auf_vielfaches_von_8} % Nr. 1


\paragraph{Aufrunden auf Vielfaches von 8 (Abb. \ref{fig:aufrunden_auf_vielfaches_von_8})}

\includegraphicsdeluxe{aufrunden_auf_vielfaches_von_8.jpg}{Aufrunden auf Vielfaches von 8 (binär)}{fig:aufrunden_auf_vielfaches_von_8} % Nr. 2

\section{Verbesserung der Freispeicherverwaltung}

\subsection{ganz einfach}
\begin{itemize}
\item malloc(n) nimmt die nächsten n Byte $((n+7)8 \sim 7)$ vom Poolsegment.
\item free(p) wird nicht implementiert 
\end{itemize}

\subsection{einfach für eine feste Größe N} Verwalten einer Liste mit freien Blöcken.

\paragraph{malloc(N)}
\begin{itemize}
\item wenn Freiliste leer N Bytes vom Poolsegment
\item sonst erstes Element der Freiliste
\end{itemize}

\paragraph{free(p)} Node p in die Freiliste einfügen.

Das ist einfach für eine Variable n. Die Knoten sehen so aus: Abb. \ref{fig:freiliste_malloc}
\paragraph{malloc(n):} n wir auf ein Vielfaches von 8 aufgerundet. In der Freiliste wird nach einem \textbf{passenden} Element gesucht. Wenn nicht, wird aus dem Poolsegment n+8 Byte (8 entspricht der Größe size=)genommen. Size wird eingetragen und Adresse vom OCTA nach size zurückgegeben. 
\paragraph{Was heißt passend?} Es gibt unterschiedliche Strategien. 
\begin{itemize}
\item First Fit: den ersten der passt
\item Best Fit: den der als bester passt
\end{itemize}
\includegraphicsdeluxe{freiliste_malloc.jpg}{So könnte die Freiliste aussehen}{fig:freiliste_malloc} % Nr. 

\paragraph{Best Fit und First Fit} 
Beide Strategien können mit dem Teilen von Knoten kombiniert werden (Abb. \ref{fig:geteilteKnoten}). Teilen in der Regel nur, wenn der Rest eine gewisse Mindestgröße hat. Problem: Speicherfragmentierung. 
\includegraphicsdeluxe{geteilteKnoten.jpg}{geteilte Knoten}{fig:geteilteKnoten} % Nr. 4
\paragraph{Beispiel:} Knoten zwischen 16 und 1600 Byte zufällig gemischt, First Fit, Anforderung von 64 Byte.

\subsection{Verbesserungen}
\begin{itemize}
\item Durchsuchen der Freiliste nicht jedes Mal von vorne, sondern von der letzten Fundstelle aus. 
\item Zusammenfügen von Knoten beim Freigeben. Nachteil: Durchsuchen der Liste dauert ggf. lange. Man kann das verkürzen, indem man die Liste nach Adressen sortiert. 
\item Garbage Collection
	\subitem Es erfolgt keine Freigabe
	\subitem stattdessen gibt es ein Verzeichnis aller Zeiger (auf dem Stack, z.B. in Java).
	\subitem Hier gibt es einen Mark and Sweep Algorithmus. Man beginnt mit den Zeigern auf dem Stack und markiert sie , dann markiert man alle Objekte, auf die sie zeigen und alle Zeiger in den Objekten und so weiter (Mark). Nun wird der Speicher durchlaufen und nicht-markierte OElemente freigegeben (Sweep). 
\end{itemize}


\end{document}