% Vorlesung vom 27.05.2015
\renewcommand{\ldate}{2015-05-27}	% define lessiondate
\fancyfoot[R]{\tiny{SS 2015 / \ldate}}


\subsection{Fortsetzung Synchronisation von Threads und Prozessen}

\paragraph{CSWAP Instruktion (Compare and Swap):}

\begin{lstlisting}
CSWAP $X,$Y,$Z	% adresse = $Y+$Z
\end{lstlisting}

Vergleiche M[adresse] mit rP. \marginpar{rP: Prediction register}
Falls gleich:  
\begin{lstlisting}
speichern: M[adresse] $ <-- $ $X
setze: $X <-- 1
\end{lstlisting}
sonst:
\begin{lstlisting}
lade: rP <-- M[adresse]
setze: $x <-- 0
\end{lstlisting}
Die Instruktion ist atomar, d.h. \textbf{ununterbrechbar}.

\textbf{Gemeinsame Daten}

\begin{figure}[htbp]
% Nr. 1
\includegraphics[width=1\textwidth]{pics/\ldate_gemeinsame_daten.jpg}
\caption{Gemeinsame Daten}
\label{fig:gemeinsame_daten}
\end{figure}

\begin{lstlisting}
Aquire:	falls S == 0, setze S auf 1 (blocking)
	bearbeite buffer (protected code)

Release: setze S auf 0
\end{lstlisting}
Was nicht geht:
\begin{lstlisting}
1H	LDO $0,S
	BNZ $0,1B	
	STCO 1,S
\end{lstlisting}

\textbf{mit CSWAP}
\begin{lstlisting}
1H	PUT rP,0
	SET $0,1
	CSWAP $0,S	% atomare Operation
	BZ $0,1B
\end{lstlisting}


\subsection{Strategien zum Schreiben in den Cache}
\begin{figure}[htbp]
% Nr. 2
\includegraphics[width=1\textwidth]{pics/\ldate_speicherhirarchie.jpg}
\caption{Speicherhirarchie in der CPU}
\label{fig:speicherhirarchie}
\end{figure}
\marginpar{ganz viele Erklärungen von Herrn Ruckert, die v.a. parallel zum Abmalen des Tafelbilds nicht aufgeschrieben werdne können.}

\paragraph{Write allocate} Die passende cacheline wird geladen, der Wert kommt in die cacheline.
\paragraph{Write Back} Die passende cacheline wird geladen, geändert und geschrieben.
\paragraph{Write through} Falls die passende cacheline vorhanden ist, wie write back, sonst wird der Wert am cache vorbei in die nächsttiefere Speicherebene geschrieben (vgl. Abb. \ref{fig:speicherhirarchie}).

\subsection{Funktionsweise von Caches}
Caches dienen dem schnellen Speicherzugriff. Meist sind sie direkt am Chip (L1-Caches). L2-Caches sind im gleichen Gehäuse der CPU. Es gibt sogenannte cachelines (z.B. 64 Byte). Es wird immer die komplette cacheline geladen oder gespeichert, nie einzelne Bytes (Abb. \ref{fig:cachlines}).  

\begin{figure}[htbp]
% Nr. 3
\includegraphics[width=1\textwidth]{pics/\ldate_cachelines.jpg}
\caption{Cachelines}
\label{fig:cachlines}
\end{figure}

\subsection{Non Blocking Synchronisation}
\paragraph{wait-free:} Jede Operation wird in endlich vielen Schritten fertig. 
\paragraph{lock-free:} Irgendeine Operation wird in endlich vielen Schritten fertig. 
\paragraph{obstruction-free:} Ein Thread wird in endlich vielen Schritten fertig, wenn alle anderen Threads gerade pausieren. 

\subsection{Cacheprotokolle MESI}
\begin{figure}[htbp]
% Nr. 4
\includegraphics[width=1\textwidth]{pics/\ldate_cache_mesi.jpg}
\caption{Funktionsweise von Cache MESI}
\label{fig:cache_mesi}
\end{figure}

\section{Datenstrukturen}
Bisher haben wir z.B. Arrays kennengelernt. Jetzt schauen wir uns dynamische Datenstrukturen an. Das sind Datenstrukturen, die zur Laufzeit die Größe ändern. 
\paragraph{Dynamische Speicherverwaltung} z.B. die Funktion malloc(size) in C.\marginpar{memmory allocate} Diese Funktion gibt die Adresse zurück, an der size BYTE verfügbar sind. Die Funktion free(adresse) gibt den Speicher an der angegebenen Adresse wieder frei. 

\subsection{Verkettete Listen} 

\begin{figure}[htbp]
% Nr. 5
\includegraphics[width=1\textwidth]{pics/\ldate_verkettete_listen.jpg}
\caption{verkettete Listen}
\label{fig:verkettete_listen}
\end{figure}

\paragraph{Insert:} Verkettete Liste mit NEXT (Adresse des nächsten Knotens) und VALUE (Wert 1 OCTA). 

\begin{figure}[htbp]
% Nr. 6
\includegraphics[width=1\textwidth]{pics/\ldate_llist_insert.jpg}
\caption{insert bei verketteten Listen}
\label{fig:llist_insert}
\end{figure}

\begin{lstlisting}
	PREFIX :Insert:
Head IS $0	% Adresse des list heads (param 1)
NewElement IS $1	% Adresse des neuen Elements (param 2)
tmp IS $2

:Insert
	LDOU tmp,Head,0
	STOU NewElement,Head,0
	STOU tmp,NewElement,0
	POP 0,0
\end{lstlisting}

Variante 2
\begin{lstlisting}
	PREFIX :Insert:
Head IS $0	% Adresse des list heads (param 1)
NewElement IS $1	% Adresse des neuen Elements (param 2)
tmp IS $2

:Insert
	LDOU tmp,Head,0
	STOU tmp,NewElement,0
	STOU NewElement,Head,0
	
	POP 0,0
\end{lstlisting}

\paragraph{Delete:} 

\begin{lstlisting}
	PREFIX :Delete:
Head IS $0	% Adresse des list heads (@param 1)
% Rueckgabewert: 
%	Null: wenn die Liste leer ist
%	sonst: Adresse des geloeschten Elements
return IS $1
tmp IS $2

:Delete LDOU return,Head,0
	BZ return,1F
	LDOU tmp,return,0
	STOU tmp,Head,0
1H	SET $0,return
	POP 1,0
\end{lstlisting}