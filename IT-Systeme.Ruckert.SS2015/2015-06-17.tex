% Vorlesung vom 17.06.2015
\renewcommand{\ldate}{2015-06-17}	% define lessiondate
\fancyfoot[R]{\tiny{SS 2015 / \ldate}}

\section{Fragen von Studenten. Antworten vom Professor}

\subsection{Wie funktioniert loop unrolling?}

\begin{lstlisting}
	PREFIX :InsertEnd:
new	IS $0	Adresse des neuen Knoten
head	IS $1	Adresse von der Adresse des ersten Knotens
next	IS $2
NEXT	IS 0	Offset des NEXT-Feldes

% Variante 1
1H	SET head,next
:InsertEnd	LDOU	next,head,NEXT
	BNZ	next,1B
	STOU	new,head,NEXT
	POP	0,0
	
% Variante 2	
1H	SET head,next
:InsertEnd	LDOU	next,head,NEXT
	BZ	next,0F
	SET	head,next
	LDOU	next,head,NEXT
	BNZ	next,1B
	STOU	new,head,NEXT
	POP	0,0
	
OH	STOU	new,head,NEXT	
\end{lstlisting}

\includegraphicsdeluxe{foto_optimierung_code.jpg}{Optimierung mittels loop doubleing; Rot: neuer Wert, Gelb: Verwendung}{fig:foto_optimierung_code} % Nr. FOTO!

\paragraph{copy propagation} Statt der Kopie verwendet man das Original. Der optimierte Code lautet dann:

\begin{lstlisting}
:InsertEnd LDOU	next,head,NEXT
	BZ	next,0F

	LDOU	head,next,NEXT
	PBNZ	head,:InsertEnd
	STOU	new,next,NEXT
	POP	0,0
	
0H	STOU	new,head,NEXT
	POP	0,0	
\end{lstlisting} 

\subsection{Wie kann man die Anzahl von Bytes auf ein Vielfaches von 8 aufrunden?}

\subsubsection{Wie geht das Dezimal?}
\begin{itemize}
\item Abrunden: $1234567 \rightarrow 1234000$
\item Aufrunden: $1234567 \rightarrow 12345000$
\item Aufrunden durch Abrunden: Erst $+999$, dann abrunden: $1234557 + 999 = 12345566 \rightarrow 12345000$
\end{itemize}

\subsubsection{Wie geht das binär?}
Abrunden auf Vielfaches von 8 (Abb. \ref{fig:abrunden_auf_vielfaches_von_8})
\begin{lstlisting}
AND	size,size,7
(AND	size,size,#FFF....F8 % wg. Groesse nicht moeglich)
ANDN	size,size,7 % Ersatz fuer Zeile oben drueber
\end{lstlisting}
\includegraphicsdeluxe{abrunden_auf_vielfaches_von_8.jpg}{Abrunden auf Vielfaches von 8 (binär)}{fig:abrunden_auf_vielfaches_von_8} % Nr. 1


\paragraph{Aufrunden auf Vielfaches von 8 (Abb. \ref{fig:aufrunden_auf_vielfaches_von_8})}

\includegraphicsdeluxe{aufrunden_auf_vielfaches_von_8.jpg}{Aufrunden auf Vielfaches von 8 (binär)}{fig:aufrunden_auf_vielfaches_von_8} % Nr. 2

\section{Verbesserung der Freispeicherverwaltung}

\subsection{ganz einfach}
\begin{itemize}
\item malloc(n) nimmt die nächsten n Byte $((n+7)8 \sim 7)$ vom Poolsegment.
\item free(p) wird nicht implementiert 
\end{itemize}

\subsection{einfach für eine feste Größe N} Verwalten einer Liste mit freien Blöcken.

\paragraph{malloc(N)}
\begin{itemize}
\item wenn Freiliste leer N Bytes vom Poolsegment
\item sonst erstes Element der Freiliste
\end{itemize}

\paragraph{free(p)} Node p in die Freiliste einfügen.

Das ist einfach für eine Variable n. Die Knoten sehen so aus: Abb. \ref{fig:freiliste_malloc}
\paragraph{malloc(n):} n wir auf ein Vielfaches von 8 aufgerundet. In der Freiliste wird nach einem \textbf{passenden} Element gesucht. Wenn nicht, wird aus dem Poolsegment n+8 Byte (8 entspricht der Größe size=)genommen. Size wird eingetragen und Adresse vom OCTA nach size zurückgegeben. 
\paragraph{Was heißt passend?} Es gibt unterschiedliche Strategien. 
\begin{itemize}
\item First Fit: den ersten der passt
\item Best Fit: den der als bester passt
\end{itemize}
\includegraphicsdeluxe{freiliste_malloc.jpg}{So könnte die Freiliste aussehen}{fig:freiliste_malloc} % Nr. 

\paragraph{Best Fit und First Fit} 
Beide Strategien können mit dem Teilen von Knoten kombiniert werden (Abb. \ref{fig:geteilteKnoten}). Teilen in der Regel nur, wenn der Rest eine gewisse Mindestgröße hat. Problem: Speicherfragmentierung. 
\includegraphicsdeluxe{geteilteKnoten.jpg}{geteilte Knoten}{fig:geteilteKnoten} % Nr. 4
\paragraph{Beispiel:} Knoten zwischen 16 und 1600 Byte zufällig gemischt, First Fit, Anforderung von 64 Byte.

\subsection{Verbesserungen}
\begin{itemize}
\item Durchsuchen der Freiliste nicht jedes Mal von vorne, sondern von der letzten Fundstelle aus. 
\item Zusammenfügen von Knoten beim Freigeben. Nachteil: Durchsuchen der Liste dauert ggf. lange. Man kann das verkürzen, indem man die Liste nach Adressen sortiert. 
\item Garbage Collection
	\subitem Es erfolgt keine Freigabe
	\subitem stattdessen gibt es ein Verzeichnis aller Zeiger (auf dem Stack, z.B. in Java).
	\subitem Hier gibt es einen Mark and Sweep Algorithmus. Man beginnt mit den Zeigern auf dem Stack und markiert sie , dann markiert man alle Objekte, auf die sie zeigen und alle Zeiger in den Objekten und so weiter (Mark). Nun wird der Speicher durchlaufen und nicht-markierte OElemente freigegeben (Sweep). 
\end{itemize}
