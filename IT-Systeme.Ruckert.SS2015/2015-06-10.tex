% Vorlesung vom 10.06.2015
\renewcommand{\ldate}{2015-06-10}	% define lessiondate
\fancyfoot[R]{\tiny{SS 2015 / \ldate}}

\section{Verwaltung des dynamischen Speichers zur Laufzeit}
Bei Objekten gibt es Konstruktoren und Destruktoren. In Sprachen wie C gibt es malloc, free. Im Assembler gibt es meistens ein Pool Segment (Intel: eXtra Segment). Hier wird ein sogenannter Heap eingerichtet, der Speicher reserviert und freigibt. \\
Wird ein Programm mit Argumenten gestartet stehen in \$0 die Anzahl der Argumente und in \$1 4...8 die Adresse des ersten Arguments. Die Werte der Argumente selbst stehen im Pool Segment 0x4000...0.

\subsection{Im Poolsegment dynmaisch Speicher anfordern}
Die Funktion \textbf{malloc}\index{malloc} soll erstellt werden (Parameter: Anzahl BYTE, Rückgabewert: Adresse wo diese BYTE verfügbar sind.):
\begin{itemize}
\item Rückgabe wert = erstes OCTA im Poolsegment.
\item Anzahl BYTES auf erstes OCTA aufaddieren
\item Anzahl BYTES vorher auf ein Vielfaches von acht aufrunden (wegen Alignment\index{Alignment}).
\end{itemize}

\begin{lstlisting}
	LOC #100
tmp	IS	$0
Main	SET	tmp+1,22
	PUSHJ	tmp,Malloc
	STCO	22,tmp
	SET	tmp+1,8
	PUSHJ	tmp,Malloc
	STCO	8,tmp,0
	TRAP	0,Halt,0
	
	PREFIX :Malloc:		% Version 1
size	IS	$0
base	IS	$1
free	IS	$2
tmp		IS	$3
:Malloc SETH	base,#4000
	LDOU	free,base,0
	ADD	size,size,7
	ANDN	size,size,7
	ADDU	tmp,free,size
	STOU	tmp,base,0
	SET	$0,free 
	POP	1,0	
\end{lstlisting}

\subsection{Beispielanwendung}
Wir speichern Nodes die aus zwei OCTA bestehen, einem Zeiger NEXT und einem Wert VALUE. Wir wollen zwei Funktionen New und Old haben. \textbf{Old} hat einen Parameter (Knoten) und fügt den Knoten der Liste mit alten Knoten hinzu. \textbf{New} prüft, ob alte Knoten vorhanden sind und wenn ja, ob diese genutzt werden können. Sonst nutzt man Malloc und reserviert sich so neuen Speicher für ein neuen Knoten (Node). 

\begin{lstlisting}
	LOC Data_Segment
	GREG	@
OldNodes	OCTA	0	% derzeit leer

	LOC #100
tmp	IS	$0
NEXT	IS	0
VALUE 	IS	8
Main	PUSHJ	tmp,New
	STCO	22,tmp,VALUE
	
	PUSHJ	tmp,New	
	STCO	8,tmp,VALUE
	
	SET	tmp+1,tmp
	PUSHJ	tmp,Old
	
	PUSHJ	tmp,New
	STCO	200,tmp,VALUE
	
	TRAP	0,Halt,0	% Ende Gelaende
	
	PREFIX :Malloc:		% Version 1
size	IS	$0
base	IS	$1
free	IS	$2
tmp		IS	$3
:Malloc SETH	base,#4000
	LDOU	free,base,0
	ADD	size,size,7
	ANDN	size,size,7
	ADDU	tmp,free,size
	STOU	tmp,base,0
	SET	$0,free 
	POP	1,0	
	
	PREFIX :Old:
Node	IS	$0	% Parameter
First	IS	$1

:Old	LDOU	First,:OldNodes % laedt Zeiger auf alten Knoten
	STOU	First,Node,:NEXT
	STOU	Node,:OldNodes
	POP	0,0
	
	PREFIX :New:
rJ	IS	$0
First	IS	$1	
tmp	IS	$2
:New	LDOU	First,:OldNodes
	BZ	First,1F	% nichts drin
	
	LDOU	tmp,First,:NEXT	% laden des Nextpointers vom 1. Knoten
	STOU	tmp,:OldNodes	% speichern
	SET	$0,First
	POP	1,0
	
1H	GET	rJ,:rJ
	SET	tmp+1,16	% Parameter Groesse = 10 uebergeben
	PUSHJ	tmp,:Malloc
	PUT	:rJ,rJ
	SET	$0,tmp
	POP	1,0

\end{lstlisting}

\subsection{Verschiedene Funktionen auf verketteten Listen}

\subsubsection{Einfügen am Anfang}
Haben wir bereits betrachtet.

\subsubsection{Löschen am Anfang}
Haben wir bereits betrachtet.

\subsubsection{Einfügen am Ende}

\paragraph{Version 0}
\begin{lstlisting}
:insertEnde LDOU	last,Head,0
0H	LDOU	tmp,last,:NEXT
	BZ	tmp,1F
	
	SET	last,tmp
	JMP	0B
	
1H	STOU	Node,Last,:NEXT
	POP	0,0	
\end{lstlisting}

\paragraph{Version 1}
\begin{lstlisting}
:insertEnde SET	last,Head
0H	LDOU	tmp,last,:NEXT
	BZ	tmp,1F
	
	SET	last,tmp
	JMP	0B
	
1H	STOU	Node,Last,:NEXT
	POP	0,0	
\end{lstlisting}

\paragraph{Version 2}
\begin{lstlisting}
:insertEnde	LDOU	tmp,Head,:NEXT
	BZ	tmp,1F
	
	SET	Head,tmp
	JMP	:InsertEnde
	
1H	STOU	Node,Head,:NEXT
	POP	0,0	
\end{lstlisting}

\paragraph{Version 3}
\begin{lstlisting}
Nodes	IS	$0
Head	IS	$1
tmp	IS	$2

:InsertEnd	LDOU	tmp,Head,:NEXT
	BZ	tmp,1F
	SET	Head,tmp
	JMP	:InsertEnd
1H	STOU	Node,Head,:NEXT
	POP	0,0	
\end{lstlisting}

\paragraph{Version 4}
\begin{lstlisting}
Nodes	IS	$0
Head	IS	$1
tmp	IS	$2
1H	SET	Head,tmp
:InsertEnd	LDOU	tmp,Head,:NEXT
	PBNZ	tmp,1B
	STOU	Node,Head,:NEXT
	POP	0,0	
\end{lstlisting}

\paragraph{Version 5 mit Loop doubling}
\begin{lstlisting}
Nodes	IS	$0
Head	IS	$1
tmp	IS	$2
1H	SET	Head,tmp
:InsertEnd	LDOU	tmp,Head,:NEXT
	PBNZ	tmp,1B
	STOU	Node,Head,:NEXT
	POP	0,0	
\end{lstlisting}
